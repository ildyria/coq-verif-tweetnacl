%XXX-Peter: shouldn't verifying fixed-length for loops be rather standard?
%XXX Benoit: it is simple if the argument is increasing or if the "recursive call"
% is made before the computations.
% This is not the case here: you compute idx 255 before 254...

% Can we shorten the next paragraph?
\subheading{Verifying \texttt{for} loops.}
Final states of \texttt{for} loops are usually computed by simple recursive functions.
However, we must define invariants which are true for each iteration step.

Assume that we want to prove a decreasing loop where indexes go from 3 to 0.
Define a function $g : \N \rightarrow State  \rightarrow State $ which takes as
input an integer for the index and a state, then returns a state.
It simulates the body of the \texttt{for} loop.
Define the recursion: $f : \N \rightarrow State \rightarrow State $ which
iteratively applies $g$ with decreasing index:
\begin{equation*}
  f ( i , s ) =
  \begin{cases}
  s & \text{if } s = 0 \\
  f( i - 1 , g ( i - 1  , s )) & \text{otherwise}
  \end{cases}
\end{equation*}
Then we have:
\begin{align*}
  f(4,s) &= g(0,g(1,g(2,g(3,s))))
\end{align*}
To prove the correctness of $f(4,s)$, we need to prove that intermediate steps
$g(3,s)$; $g(2,g(3,s))$; $g(1,g(2,g(3,s)))$; $g(0,g(1,g(2,g(3,s))))$ are correct.
Due to the computation order of recursive function, our loop invariant for
$i\in\{0,1,2,3,4\}$ cannot use $f(i)$.
To solve this, we define an auxiliary function with an accumulator such that
given $i\in\{0,1,2,3,4\}$, it will compute the first $i$ steps of the loop.

We then prove for the complete number of steps, the function with the accumulator
and without returns the same result.
We formalized this result in a generic way in Appendix~\ref{subsubsec:for}.

Using this formalization, we prove that the 255 steps of the Montgomery ladder
in C provide the same computations as in \coqe{RFC}.
% %
