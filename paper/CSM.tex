\documentclass{article}
\usepackage{geometry}
 \geometry{
 a4paper,
 total={190mm,257mm},
 left=10mm,
 top=20mm,
 }
\usepackage{epsfig}
\usepackage{setup}
\newtheorem{theorem}{Theorem}[section]
\newtheorem{lemma}[theorem]{Lemma}
\newtheorem{proposition}[theorem]{Proposition}
\newtheorem{corollary}[theorem]{Corollary}

\newcommand\invisiblesection[1]{%
  \refstepcounter{section}
\sectionmark{#1}}

\newenvironment{proof}[1][Proof]{\begin{trivlist}
\item[\hskip \labelsep {\bfseries #1}]}{\end{trivlist}}
  \newenvironment{definition}[1][Definition]{\begin{trivlist}
\item[\hskip \labelsep {\bfseries #1}]}{\end{trivlist}}
  \newenvironment{example}[1][Example]{\begin{trivlist}
\item[\hskip \labelsep {\bfseries #1}]}{\end{trivlist}}
  \newenvironment{remark}[1][Remark]{\begin{trivlist}
\item[\hskip \labelsep {\bfseries #1}]}{\end{trivlist}}

  \newcommand{\qed}{\nobreak \ifvmode \relax \else
    \ifdim\lastskip<1.5em \hskip-\lastskip
    \hskip1.5em plus0em minus0.5em \fi \nobreak
  \vrule height0.75em width0.5em depth0.25em\fi}


% \DeclareMathOperator*{\argmin}{arg\,min}
% \DeclareMathOperator*{\Vol}{Vol}
% \DeclareMathOperator*{\Supp}{Supp}
% \newcommand{\cR}{\ensuremath{\mathcal R}}
% \newcommand{\cS}{\ensuremath{\mathcal S}}
%
% \newcommand{\CVPDf}{\ensuremath{\algo{CVP}_{\tilde{D}_4}}}
% \newcommand{\CVPDfZf}{\ensuremath{\algo{CVP}_{D_4/\ZZ^4}}}
% \newcommand{\LDEncode}{\algo{Encode}}
% \newcommand{\LDDecode}{\algo{Decode}}



\usepackage{soul}\let\strikethrough\st\let\st\undefined

% \newcommand{\T}{L}

\newcommand{\Z}{\ensuremath{\mathbb{Z}}}
\newcommand{\K}{\ensuremath{\mathbb{K}}}
\newcommand{\ZZ}{\ensuremath{\mathbb{Z}}}
\newcommand{\corr}{\,\hat{=}\,}
\newcommand{\B}{\ensuremath{\mathbb{B}}}
\newcommand{\E}{\ensuremath{\mathbb{E}}}
\newcommand{\Oinf}{\ensuremath{\mathcal{O}}}
\newcommand{\F}[1]{\ensuremath{\mathbb{F}_{#1}}}

% \newcommand{\RR}{\ensuremath{\mathbb{R}}}
\newcommand{\N}{\ensuremath{\mathbb{N}}}

\newcommand{\mbf}{\ensuremath{\mathbf}}
% \newcommand{\rando}{\ensuremath{\xleftarrow{\$}}}
% \newcommand{\la}{\ensuremath{\leftarrow}}

% \newcommand{\MD}{\ensuremath{\mathcal{D}}\xspace}
% \newcommand{\MS}{\ensuremath{\mathcal{D}}\xspace}
% \newcommand{\MA}{\ensuremath{\mathcal{A}}\xspace}
% \newcommand{\MB}{\ensuremath{\mathcal{B}}\xspace}
% \newcommand{\MP}{\ensuremath{\mathcal{P}}\xspace}
% \newcommand{\tA}{\ensuremath{t_{\mathcal{A}}}}
% \newcommand{\eA}{\ensuremath{\varepsilon_{\mathcal{A}}}}
% \newcommand{\tS}{\ensuremath{t_{\mathcal{D}}}}
% \newcommand{\eS}{\ensuremath{\varepsilon_{\mathcal{D}}}}

% \newcommand{\vX}{\mathcal X}
% \newcommand{\vY}{\mathcal Y}


% \newcommand{\MR}{\ensuremath{\mathcal{R}}\xspace}
% \newcommand{\tR}{\ensuremath{t_{\mathcal{R}}}}
% \newcommand{\eR}{\ensuremath{\varepsilon_{\mathcal{R}}}}
%
% \newcommand{\tD}{\ensuremath{t_{\mathcal{D}}}}
% \newcommand{\eD}{\ensuremath{\varepsilon_{\mathcal{D}}}}
% \newcommand{\sis}{\textsf{SIS}\xspace}
% \newcommand{\isis}{\textsf{ISIS}\xspace}
% \newcommand{\lwe}{\textsf{LWE}\xspace}
% \newcommand{\dlwe}{\textsf{LWE}\xspace}
% \newcommand{\rdlwe}{\textsf{R-LWE}\xspace}
%
% \newcommand{\ee}{\ensuremath{\varepsilon}}
%
% \newcommand{\sS}{\ensuremath{\sigma}}
% \newcommand{\sE}{\ensuremath{\sigma}}
% \newcommand{\DS}{\ensuremath{D_{\sS}}}
% \newcommand{\DE}{\ensuremath{D_{\sE}}}
% %\newcommand{\Dy}{\ensuremath{D_y}}
% \newcommand{\Dy}{\ensuremath{[-B, B]}}
% \newcommand{\Dysc}{\ensuremath{D_{y,\mat{Sc}}}}
% \newcommand{\Dz}{\ensuremath{D_z}}
% \newcommand{\bit}{\ensuremath{\{0,1\}}}
% \newcommand{\eqdef}{\stackrel{\mathrm{def}}=}
% \newcommand{\rand}{\getsr}

% \newcommand{\rounddq}[1]{\ensuremath{\lfloor #1 \rceil_{d,q}}}
% \newcommand{\roundd}[1]{\ensuremath{\lfloor #1 \rceil_d}}
% \newcommand{\norm}[1]{\ensuremath{||#1||}}

% \newcommand{\KeyGen}{\keygen}
% \newcommand{\Sign}{\sign}
% \newcommand{\Verify}{\verify}
% \newcommand{\keygen}{\ensuremath{\mathsf{KeyGen}}}
% \newcommand{\sign}{\ensuremath{\mathsf{Sign}}}
% \newcommand{\verify}{\ensuremath{\mathsf{Verify}}}
% \newcommand{\start}{\underline{\ensuremath{\mathsf{Start}}}\xspace}
% \newcommand{\randO}{\ensuremath{\underline{\mathsf{Rand}}^{O_c}}\xspace}
% \newcommand{\signO}{\ensuremath{\underline{\mathsf{Sign}}^{O_c}}\xspace}
% \newcommand{\finishO}{\ensuremath{\underline{\mathsf{Finish}}^{O_c}}}
% \newcommand{\instance}{\underline{\ensuremath{\mathsf{Instance}}}}
% \newcommand{\Sample}{\ensuremath{\mathsf{Sample}}}

% \newcommand{\keygenQ}{\ensuremath{\mathsf{KeyGen}}}
% \newcommand{\signQ}{\ensuremath{\mathsf{Sign}}}
% \newcommand{\verifyQ}{\ensuremath{\mathsf{Verify}}}
% \newcommand{\CHgen}{\ensuremath{\mathsf{ChGen}}}

% \newcommand{\inputtext}{\textsf{INPUT:}\;}
% \newcommand{\outputtext}{\textsf{OUTPUT:}\;}
% \newcommand{\iftext}{\mathbf{if\;}}
% \newcommand{\elsetext}{\mathbf{else\;}}
% \newcommand{\then}{\mathbf{then\;}}
% \newcommand{\return}{\mathbf{return\;}}

% \newcommand{\mat}[1]{\mathbf{#1}}
% \renewcommand{\vec}[1]{\mathbf{#1}}
% \newcommand{\modq}{\ensuremath{\; (\bmod \; q)}}
%\newcommand{\modq}{\ensuremath{\; (q)}}
% \newcommand{\ip}[2]{\ensuremath{\langle {#1},{#2}\rangle}}

% \newcommand{\adv}{\ensuremath{\mathcal{A}}}
% \newcommand{\secpar}{\ensuremath{\lambda}}

% \newcommand{\sk}{\ensuremath{\mathsf{sk}}\xspace}
% \newcommand{\pk}{\ensuremath{\mathsf{vk}}\xspace}
% \newcommand{\sst}{\ensuremath{\mathsf{st}}\xspace}

% \newcommand{\CH}{\ensuremath{\mathcal{CH}}}
% \newcommand{\ch}{\ensuremath{\mathsf{CH}}}
% \newcommand{\Mek}{\ensuremath{\mathsf{M_{ek}}}}
% \newcommand{\Yek}{\ensuremath{\mathsf{Y_{ek}}}}
% \newcommand{\Rek}{\ensuremath{\mathsf{R_{ek}}}}
% \newcommand{\ek}{\ensuremath{\mathsf{ek}}}
% \newcommand{\td}{\ensuremath{\mathsf{td}}}
% \newcommand{\Gen}{\ensuremath{\mathsf{Gen}}}

% \newcommand{\negl}{\ensuremath{\mathsf{negl}}}


% \newcommand{\R}{\mathcal{R}}
% \newcommand{\Rp}{\mathcal{R}_p}
% \newcommand{\Rq}{\mathcal{R}_q}
% \newcommand{\Rqk}[1]{\mathcal{R}_{q,#1}}

% \def\getsr{\stackrel{{\scriptscriptstyle \$}}{\leftarrow}}
% \def\gets{{\leftarrow}}


% \newcommand{\Uniform}{\mathcal{U}}
% \newcommand{\UniRk}[1]{\Rqk{#1}}
% \newcommand{\UniRq}{\Rq}


\newcommand{\zeropo}{{\bf 0}}
\newcommand{\Apo}{{\bf A\xspace}}
\newcommand{\Bpo}{{\bf B\xspace}}
\newcommand{\Ipo}{{\bf I\xspace}}
\newcommand{\apo}{{\bf a\xspace}}
\newcommand{\bpo}{{\bf b\xspace}}
\newcommand{\cpo}{{\bf c\xspace}}
\newcommand{\Cpo}{{\bf C\xspace}}
\newcommand{\dpo}{{\bf d\xspace}}
\newcommand{\epo}{{\bf e\xspace}}
\newcommand{\ppo}{{\bf p\xspace}}
\newcommand{\fpo}{{\bf f\xspace}}
\newcommand{\gpo}{{\bf g\xspace}}
\newcommand{\Gpo}{{\bf G\xspace}}
\newcommand{\hpo}{{\bf h\xspace}}
\newcommand{\kpo}{{\bf k\xspace}}
\newcommand{\Kpo}{{\bf K\xspace}}
\newcommand{\lpo}{{\bf l\xspace}}
\newcommand{\Lpo}{{\bf L\xspace}}
\newcommand{\mpo}{{\bf m\xspace}}
\newcommand{\rpo}{{\bf r\xspace}}
\newcommand{\spo}{{\bf s\xspace}}
\newcommand{\tpo}{{\bf t\xspace}}
\newcommand{\upo}{{\bf u\xspace}}
\newcommand{\Upo}{{\bf U\xspace}}
\newcommand{\vpo}{{\bf v\xspace}}
\newcommand{\wpo}{{\bf w\xspace}}
\newcommand{\xpo}{{\bf x\xspace}}
\newcommand{\ypo}{{\bf y\xspace}}
\newcommand{\zpo}{{\bf z\xspace}}
\newcommand{\Zpo}{{\bf Z\xspace}}

\newcommand{\Spo}{{\bf S}}
\newcommand{\Ypo}{{\bf Y}}

% \newcommand{\algo}[1]{\textsf{#1}}

% \newcommand{\PeikertDBL}{\algo{dbl}}
% \newcommand{\PeikertREC}{\algo{rec}}


% \newcommand{\PowMul}{{\text{\sf PowMul}}}
% \newcommand{\PowMulPsi}{{\text{\sf PowMul}_{\psi}}}
\newcommand{\ie}{{\textit{i.e.}}\;}
\newcommand{\eg}{{\textit{e.g.}}\;}
\newcommand{\p}{\ensuremath{2^{255}-19}}
\newcommand{\Zfield}{\ensuremath{\mathbb{Z}_{\p}}}
\newcommand{\Ffield}{\ensuremath{\mathbb{F}_{\p}}}


\begin{document}

Generic definition of the ladder:

\begin{lstlisting}[language=Coq]
(* Define a typeclass to encapsulate the operations *)
Class Ops (T T': Type) (Mod: T -> T):=
{
  +   : T -> T -> T;          (* Add           *)
  *   : T -> T -> T;          (* Mult          *)
  -   : T -> T -> T;          (* Sub           *)
  x^2  : T -> T;               (* Square        *)
  C_0 : T;                     (* Const 0       *)
  C_1 : T;                     (* Const 1       *)
  C_121665: T;                 (* const (a-2)/4 *)
  Sel25519: Z -> T -> T -> T;(* CSWAP         *)
  Getbit: Z -> T' -> Z;       (* ith bit       *)
}.

Fixpoint montgomery_rec (m : nat) (z : T')
  (a: T) (b: T) (c: T) (d: T) (e: T) (f: T) (x: T) :
  (* a: x2              *)
  (* b: x3              *)
  (* c: z2              *)
  (* d: z3              *)
  (* e: temporary  var  *)
  (* f: temporary  var  *)
  (* x: x1              *)
  (T * T * T * T * T * T) :=
  match m with
  | 0%nat => (a,b,c,d,e,f)
  | S n =>
      let r := Getbit (Z.of_nat n) z in
        (* k_t = (k >> t) & 1 *)
        (* swap <- k_t *)
      let (a, b) := (Sel25519 r a b, Sel25519 r b a) in
        (* (x_2, x_3) = cswap(swap, x_2, x_3) *)
      let (c, d) := (Sel25519 r c d, Sel25519 r d c) in
        (* (z_2, z_3) = cswap(swap, z_2, z_3) *)
      let e := a + c in        (* A = x_2 + z_2                     *)
      let a := a - c in        (* B = x_2 - z_2                     *)
      let c := b + d in        (* C = x_3 + z_3                     *)
      let b := b - d in        (* D = x_3 - z_3                     *)
      let d := e ^2 in          (* AA = A^2                         *)
      let f := a ^2 in          (* BB = B^2                         *)
      let a := c * a in        (* CB = C * B                      *)
      let c := b * e in        (* DA = D * A                      *)
      let e := a + c in        (* x_3 = (DA + CB)^2        --- (1/2)  *)
      let a := a - c in        (* z_3 = x_1 * (DA - CB)^2   --- (1/3)  *)
      let b := a ^2 in          (* z_3 = x_1 * (DA - CB)^2   --- (2/3)  *)
      let c := d - f in        (* E = AA - BB                     *)
      let a := c * C_121665 in (* z_2 = E * (AA + a24 * E) --- (1/3) *)
      let a := a + d in        (* z_2 = E * (AA + a24 * E) --- (2/3) *)
      let c := c * a in        (* z_2 = E * (AA + a24 * E) --- (3/3) *)
      let a := d * f in        (* x_2 = AA * BB                    *)
      let d := b * x in        (* z_3 = x_1 * (DA - CB)^2    --- (3/3) *)
      let b := e ^2 in          (* x_3 = (DA + CB)^2         --- (2/2) *)
      let (a, b) := (Sel25519 r a b, Sel25519 r b a) in
        (* (x_2, x_3) = cswap(swap, x_2, x_3) *)
      let (c, d) := (Sel25519 r c d, Sel25519 r d c) in
        (* (z_2, z_3) = cswap(swap, z_2, z_3) *)
      montgomery_rec n z a b c d e f x
    end.

Definition get_a (t:(T * T * T * T * T * T)) : T :=
match t with
  (a,b,c,d,e,f) => a
end.
Definition get_c (t:(T * T * T * T * T * T)) : T :=
match t with
  (a,b,c,d,e,f) => c
end.
\end{lstlisting}

\newpage


Instanciation of the Class \Coqe{Ops} with operations over $\Z$ and modulo \p.
\begin{lstlisting}[language=Coq]
Definition modP x := Z.modulo x (Z.pow 2 255 - 19).

(* Encapsulate in a module. *)
Module Mid.
  Definition getCarry (m:Z) : Z :=  Z.shiftr m n.
  Definition getResidue (m:Z) : Z := m - Z.shiftl (getCarry m) n.

  Definition car25519 (n:Z) : Z  :=  38 * getCarry 256 n +  getResidue 256 n.
  (* The carry operation is invariant under modulo *)
  Lemma Zcar25519_correct: forall n, n:GF = (Mid.car25519 n) :GF.

  (* Define Mid.A, Mid.M ... *)
  Definition A a b := Z.add a b.
  Definition M a b := Mid.car25519 (Mid.car25519 (Z.mul a b)).
  Definition Zub a b := Z.sub a b.
  Definition Sq a := M a a.
  Definition C_0 := 0.
  Definition C_1 := 1.
  Definition C_121665 := 121665.
  Definition Sel25519 (b p q:Z) :=   if (Z.eqb b 0) then p else q.
  Definition getbit (i:Z) (a: Z) :=
    if (Z.ltb a 0) then
      0
    else
      if (Z.ltb i 0) then
        Z.land a 1
      else
        Z.land (Z.shiftr a i) 1.
End Mid.

Definition ZPack25519 n :=
  Z.modulo n (Z.pow 2 255 - 19).

Definition Zclamp (n : Z) : Z :=
  (Z.lor (Z.land n (Z.land (Z.ones 255) (-8))) (Z.shiftl 64 (31 * 8))).

(* x^{p - 2} *)
Definition ZInv25519 (x:Z) : Z := Z.pow x (Z.pow 2 255 - 21).

(* instantiate over Z *)
Instance Z_Ops : (Ops Z Z modP) := {}.
Proof.
  apply Mid.A.        (* instantiate +       *)
  apply Mid.M.        (* instantiate *       *)
  apply Mid.Zub.      (* instantiate -       *)
  apply Mid.Sq.       (* instantiate x^2      *)
  apply Mid.C_0.      (* instantiate Const 0 *)
  apply Mid.C_1.      (* instantiate Const 1 *)
  apply Mid.C_121665. (* instantiate (a-2)/4 *)
  apply Mid.Sel25519. (* instantiate CSWAP   *)
  apply Mid.getbit.   (* instantiate ith bit *)
Defined.

(* instantiate montgomery_rec with Z_Ops *)
Definition ZCrypto_Scalarmult n p :=
  let t := montgomery_rec
    255               (* iterate 255 times *)
    (Zclamp n)        (* clamped n         *)
    1                 (* x_2                *)
    (ZUnpack25519 p)  (* x_3                *)
    0                 (* z_2                *)
    1                 (* z_3                *)
    0                 (* dummy             *)
    0                 (* dummy             *)
    (ZUnpack25519 p)  (* x_1                *) in
  ZPack25519 (Z.mul (get_a t) (ZInv25519 (get_c t))).
\end{lstlisting}

\newpage

\begin{lstlisting}[language=Coq]
Definition Crypto_Scalarmult n p :=
let t := (montgomery_rec 255
  (clamp n) Low.C_1 (Unpack25519 p) Low.C_0 Low.C_1
  Low.C_0 Low.C_0 (Unpack25519 p)) in
let a := get_a t in
let c := get_c t in
Pack25519 (Low.M a (Inv25519 c)).

Definition CSM := Crypto_Scalarmult.

Theorem Crypto_Scalarmult_Eq : forall (n p:list Z),
  Zlength n = 32 ->
  Zlength p = 32 ->
  Forall (fun x : Z, 0 <= x /\ x < 2 ^ 8) n ->
  Forall (fun x : Z, 0 <= x /\ x < 2 ^ 8) p ->
  ZofList 8 (Crypto_Scalarmult n p) =
    ZCrypto_Scalarmult (ZofList 8 n) (ZofList 8 p).
\end{lstlisting}


\end{document}
