\documentclass{article}
\usepackage{geometry}
 \geometry{
 a4paper,
 total={190mm,257mm},
 left=10mm,
 top=20mm,
 }
\usepackage{epsfig}
\usepackage{setup}

% \newtheorem{theorem}{Theorem}[section]
% \newtheorem{lemma}[theorem]{Lemma}
% \newtheorem{corollary}[theorem]{Corollary}
% \newtheorem{dfn}[theorem]{Definition}
% \newtheorem{hypothesis}[theorem]{Hypothesis}

\newcommand\invisiblesection[1]{%
  \refstepcounter{section}
\sectionmark{#1}}

% \newenvironment{proof}[1][Proof]{\begin{trivlist}
% \item[\hskip \labelsep {\bfseries #1}]}{\end{trivlist}}
%   \newenvironment{definition}[1][Definition]{\begin{trivlist}
% \item[\hskip \labelsep {\bfseries #1}]}{\end{trivlist}}
%   \newenvironment{example}[1][Example]{\begin{trivlist}
% \item[\hskip \labelsep {\bfseries #1}]}{\end{trivlist}}
%   \newenvironment{remark}[1][Remark]{\begin{trivlist}
% \item[\hskip \labelsep {\bfseries #1}]}{\end{trivlist}}

% \newcommand{\qed}{\nobreak \ifvmode \relax \else
%   \ifdim\lastskip<1.5em \hskip-\lastskip
%   \hskip1.5em plus0em minus0.5em \fi \nobreak
% \vrule height0.75em width0.5em depth0.25em\fi}

\newcommand{\sref}[1]{Section~\ref{#1}}
\newcommand{\tref}[1]{Theorem~\ref{#1}}
\newcommand{\lref}[1]{Lemma~\ref{#1}}
\newcommand{\fref}[1]{Figure~\ref{#1}}
\newcommand{\aref}[1]{Algorithm~\ref{#1}}
\newcommand{\N}{\ensuremath{\mathbb{N}}\xspace}
\newcommand{\Z}{\ensuremath{\mathbb{Z}}\xspace}
\newcommand{\K}{\ensuremath{\mathbb{K}}\xspace}
\renewcommand{\L}{\ensuremath{\mathbb{L}}\xspace}
\newcommand{\corr}{\,\hat{=}\,}
\newcommand{\Oinf}{\ensuremath{\mathcal{O}}\xspace}
\newcommand{\F}[1]{\ensuremath{\mathbb{F}_{#1}}\xspace}
\newcommand{\mbf}{\ensuremath{\mathbf}}
\newcommand{\ie}{{\textit{i.e.}},\;}
\newcommand{\eg}{{\textit{e.g.}},\;}
\newcommand{\p}{\ensuremath{2^{255}-19}}
\newcommand{\Zfield}{\ensuremath{\mathbb{Z}_{\p}}}
\newcommand{\Ffield}{\ensuremath{\mathbb{F}_{\p}}}
\newcommand{\xcoord}{$x$-coordinate\xspace}
\newcommand{\xcoords}{$x$-coordinates\xspace}


\begin{document}

Generic definition of the ladder:

\begin{lstlisting}[language=Coq]
(* Define a typeclass to encapsulate the operations *)
Class Ops (T T': Type) (Mod: T -> T):=
{
  +   : T -> T -> T;          (* Add           *)
  *   : T -> T -> T;          (* Mult          *)
  -   : T -> T -> T;          (* Sub           *)
  x^2  : T -> T;               (* Square        *)
  C_0 : T;                     (* Const 0       *)
  C_1 : T;                     (* Const 1       *)
  C_121665: T;                 (* const (a-2)/4 *)
  Sel25519: Z -> T -> T -> T;(* CSWAP         *)
  Getbit: Z -> T' -> Z;       (* ith bit       *)
}.

Fixpoint montgomery_rec (m : nat) (z : T')
  (a: T) (b: T) (c: T) (d: T) (e: T) (f: T) (x: T) :
  (* a: x2              *)
  (* b: x3              *)
  (* c: z2              *)
  (* d: z3              *)
  (* e: temporary  var  *)
  (* f: temporary  var  *)
  (* x: x1              *)
  (T * T * T * T * T * T) :=
  match m with
  | 0%nat => (a,b,c,d,e,f)
  | S n =>
      let r := Getbit (Z.of_nat n) z in
        (* k_t = (k >> t) & 1 *)
        (* swap <- k_t *)
      let (a, b) := (Sel25519 r a b, Sel25519 r b a) in
        (* (x_2, x_3) = cswap(swap, x_2, x_3) *)
      let (c, d) := (Sel25519 r c d, Sel25519 r d c) in
        (* (z_2, z_3) = cswap(swap, z_2, z_3) *)
      let e := a + c in        (* A = x_2 + z_2                     *)
      let a := a - c in        (* B = x_2 - z_2                     *)
      let c := b + d in        (* C = x_3 + z_3                     *)
      let b := b - d in        (* D = x_3 - z_3                     *)
      let d := e ^2 in          (* AA = A^2                         *)
      let f := a ^2 in          (* BB = B^2                         *)
      let a := c * a in        (* CB = C * B                      *)
      let c := b * e in        (* DA = D * A                      *)
      let e := a + c in        (* x_3 = (DA + CB)^2        --- (1/2)  *)
      let a := a - c in        (* z_3 = x_1 * (DA - CB)^2   --- (1/3)  *)
      let b := a ^2 in          (* z_3 = x_1 * (DA - CB)^2   --- (2/3)  *)
      let c := d - f in        (* E = AA - BB                     *)
      let a := c * C_121665 in (* z_2 = E * (AA + a24 * E) --- (1/3) *)
      let a := a + d in        (* z_2 = E * (AA + a24 * E) --- (2/3) *)
      let c := c * a in        (* z_2 = E * (AA + a24 * E) --- (3/3) *)
      let a := d * f in        (* x_2 = AA * BB                    *)
      let d := b * x in        (* z_3 = x_1 * (DA - CB)^2    --- (3/3) *)
      let b := e ^2 in          (* x_3 = (DA + CB)^2         --- (2/2) *)
      let (a, b) := (Sel25519 r a b, Sel25519 r b a) in
        (* (x_2, x_3) = cswap(swap, x_2, x_3) *)
      let (c, d) := (Sel25519 r c d, Sel25519 r d c) in
        (* (z_2, z_3) = cswap(swap, z_2, z_3) *)
      montgomery_rec n z a b c d e f x
    end.

Definition get_a (t:(T * T * T * T * T * T)) : T :=
match t with
  (a,b,c,d,e,f) => a
end.
Definition get_c (t:(T * T * T * T * T * T)) : T :=
match t with
  (a,b,c,d,e,f) => c
end.
\end{lstlisting}

\newpage


Instanciation of the Class \Coqe{Ops} with operations over $\ZZ$ and modulo \p.
\begin{lstlisting}[language=Coq]
Definition modP x := Z.modulo x (Z.pow 2 255 - 19).

(* Encapsulate in a module. *)
Module Mid.
  Definition getCarry (m:Z) : Z :=  Z.shiftr m n.
  Definition getResidue (m:Z) : Z := m - Z.shiftl (getCarry m) n.

  Definition car25519 (n:Z) : Z  :=  38 * getCarry 256 n +  getResidue 256 n.
  (* The carry operation is invariant under modulo *)
  Lemma Zcar25519_correct: forall n, n:GF = (Mid.car25519 n) :GF.

  (* Define Mid.A, Mid.M ... *)
  Definition A a b := Z.add a b.
  Definition M a b := Mid.car25519 (Mid.car25519 (Z.mul a b)).
  Definition Zub a b := Z.sub a b.
  Definition Sq a := M a a.
  Definition C_0 := 0.
  Definition C_1 := 1.
  Definition C_121665 := 121665.
  Definition Sel25519 (b p q:Z) :=   if (Z.eqb b 0) then p else q.
  Definition getbit (i:Z) (a: Z) :=
    if (Z.ltb a 0) then
      0
    else
      if (Z.ltb i 0) then
        Z.land a 1
      else
        Z.land (Z.shiftr a i) 1.
End Mid.

Definition ZPack25519 n :=
  Z.modulo n (Z.pow 2 255 - 19).

Definition Zclamp (n : Z) : Z :=
  (Z.lor (Z.land n (Z.land (Z.ones 255) (-8))) (Z.shiftl 64 (31 * 8))).

(* x^{p - 2} *)
Definition ZInv25519 (x:Z) : Z := Z.pow x (Z.pow 2 255 - 21).

(* instanciate over Z *)
Instance Z_Ops : (Ops Z Z modP) := {}.
Proof.
  apply Mid.A.        (* instanciate +       *)
  apply Mid.M.        (* instanciate *       *)
  apply Mid.Zub.      (* instanciate -       *)
  apply Mid.Sq.       (* instanciate x^2      *)
  apply Mid.C_0.      (* instanciate Const 0 *)
  apply Mid.C_1.      (* instanciate Const 1 *)
  apply Mid.C_121665. (* instanciate (a-2)/4 *)
  apply Mid.Sel25519. (* instanciate CSWAP   *)
  apply Mid.getbit.   (* instanciate ith bit *)
Defined.

(* instanciate montgomery_rec with Z_Ops *)
Definition ZCrypto_Scalarmult n p :=
  let t := montgomery_rec
    255               (* iterate 255 times *)
    (Zclamp n)        (* clamped n         *)
    1                 (* x_2                *)
    (ZUnpack25519 p)  (* x_3                *)
    0                 (* z_2                *)
    1                 (* z_3                *)
    0                 (* dummy             *)
    0                 (* dummy             *)
    (ZUnpack25519 p)  (* x_1                *) in
  ZPack25519 (Z.mul (get_a t) (ZInv25519 (get_c t))).
\end{lstlisting}

\newpage

\begin{lstlisting}[language=Coq]
Definition Crypto_Scalarmult n p :=
let t := (montgomery_rec 255
  (clamp n) Low.C_1 (Unpack25519 p) Low.C_0 Low.C_1
  Low.C_0 Low.C_0 (Unpack25519 p)) in
let a := get_a t in
let c := get_c t in
Pack25519 (Low.M a (Inv25519 c)).

Definition CSM := Crypto_Scalarmult.

Theorem Crypto_Scalarmult_Eq : forall (n p:list Z),
  Zlength n = 32 ->
  Zlength p = 32 ->
  Forall (fun x : Z, 0 <= x /\ x < 2 ^ 8) n ->
  Forall (fun x : Z, 0 <= x /\ x < 2 ^ 8) p ->
  ZofList 8 (Crypto_Scalarmult n p) =
    ZCrypto_Scalarmult (ZofList 8 n) (ZofList 8 p).
\end{lstlisting}


\end{document}
