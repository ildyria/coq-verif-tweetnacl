\section{Prior reviews}
\label{appendix:past-reviews}

This paper has been submitted to IEEE Symposium on Security and Privacy 2020.

\subsection{S\&P 2020 Review \#582A}

Overall merit: 3. Weak reject - The paper has flaws, but I will not argue against it.


\begin{center}
  \subheading{===== Brief paper summary (2-3 sentences) =====}
\end{center}

This paper presents a verification of the C implementation of the X25519
key-exchange protocol in the TweetNaCl library.  Correctness properties
including the C code implements the algorithm correctly are verified in Coq,
using the Verified Software Toolchain (VST).


\begin{center}
  \subheading{===== Strengths =====}
\end{center}

+ This paper produces a fully formally verified crypto library.

+ This paper presents solid technical work.


\begin{center}
  \subheading{===== Weaknesses =====}
\end{center}

- The novelty of this paper is not well explained. It's unclear whether this
paper has pushed the boundary in terms of what can be formally verified.


\begin{center}
  \subheading{===== Detailed comments for the author(s) =====}
\end{center}

I appreciate work on producing fully verified software. This paper presents
another instance: X25519 protocol. The result here is very strong: it verified
both safety properties and the correctness of the C implementation.

One main weakness of this paper is that it is hard to extract from the writing,
what are the new discoveries in this exercise. Are there new proof techniques
developed for this verification? Is there anything new about the way that loops
are handled? Is the application of reflection challenging? Are there bugs found,
in particular, in the mathematical operations?

A related comment is that it is unclear from the paper whether now we can re-use
proofs in this paper to verify  mathematical operations of other protocols that
use elliptic curve.

In the Corrections in TweetNaCl paragraph, two things were discussed. It's
unclear how and whether either one of them directly relate to the verification
effort. It would be nice if they are. Then the story for the usefulness of the
verification become stronger.

It would be nice if the authors comment on the verification effort: person month
etc. Also, did the verified version replace the version in use in production
software? If not, why not?

Section V is extremely dry. What are the high-level insights? Perhaps a picture
representation of how the theorems and lemmas fit together would be a better way
to represent this section.

In summary, this paper is in need of significant improvements in the writing and
providing detailed discussions of high-level intellectual contributions in this
verification effort.

Minor

On page 6: "each elements" => no s

On page 7, the beginning of the page, a half sentence is
dangling there.

On page 9: "each functions" => no s

\subheading{Our answer}

XXX:TODO
We corrected the typos. There are no new proof techniques in the case of this
verification. We used reflections techniques to generically solve some of our
goals such as the proof of inversion.

\subsection{S\&P 2020 Review \#582B}

Overall merit: 4. Weak accept - While flawed, the paper has merit and we should
consider accepting it.


\begin{center}
  \subheading{===== Brief paper summary (2-3 sentences) =====}
\end{center}

In this paper, the authors verify the C implementation of the X25519 elliptic
curve key exchange in the TweetNaCl library, a short implementation of the NaCl
library. The X25519 key exchange method is used in TLS 1.3, Signal, Tor, Zcash
and SSH. Within TLS 1.3 (and probably also in other protocols), any implementation
of X25519 may be used in an implementation of the standard. Thus, the present
work contributes to a fully verified code-base for TLS 1.3 (and other crucial
protocols)

The verification itself is practically significant. The paper makes a good
effort to describe the approach, but I was, at times, a little disappointed at
the description. Performing this type of verification yields substantial human
understanding and enables the authors to carry out a similar approach for other
primitives much more easily. I wish that the authors would extract their conceptual
insights on the technique and communicate those to the readers. Regardless, I
would very much like to see this work at S\&P, but I would like to encourage the
authors to de-emphasize technical details (even more) and spend more space of
the body on conceptual insights (see comments below).


\begin{center}
  \subheading{===== Detailed comments for the author(s) =====}
\end{center}

I was confused that you did not seem have found any errors in an implementation.
I can believe that the implementers of [2] are proficient implementers, but I
would have expected some minor bugs/parsing issues to be found anyway. What is
your explanation that you haven't found any issues?

p.1: You discuss [10] as using heavy annotation of code, differently from your
own code. However, as far as I understand, your Coq implementation must also be
type-annotated. Thus, I imagine that the difference lies in whether SAT-solvers
are used or not? This did not become very clear, because you mostly describe
names of tool while leaving out the details of what these tools carry out
conceptually.

p.2: Fig 1 is very nice. However, at this point, none of the notation has been
explained to the reader, and I imagine that V stands for VST? It wasn't clear to
me. Maybe, the caption of the figure could explain what the details inside the
figures represent. In particular, the Figure seems to used a nice abstraction of
properties that is not used elsewhere in the paper. I.e., elsewhere in the paper,
there is usually a lot of code details.

p.2-4: Subsection A took some mathematical background. Obviously, not all of it
could be covered, but the motivation for why certain content was presented and
not others was not clear to me as a reader, which was quite frustrating. In
Section B, it was even less clear why particular details were presented, e.g.,
the discussion of the computation of n' was completely unclear to me.
Subsection D seemed a good idea, but was very tough to follow, too much content
in short time. Subsection D seems to jump between implementation levels and
discussing high-level arguments such as Fermat's little theorem.

p.3: I very much liked the discussion of the Diff between TweetNaCl and your
modifications! I imagine that many of them are conceptually interesting in that
they make verification easier. I would have enjoyed reading a conceptual
discussion of the differences. Instead, Appendix A just contains a Diff, and as
a reader, I would need to perform the extraction of insights myself.

p.5: CSWAP: Maybe use a backward reference to p.3 where CSWAP was defined, since
in between, there was a lot of content, and relying on the reader's memory does
not really work here (or at least, it did not work for me).

p.7: I enjoyed the discussion on aliasing! This was the type of conceptual
discussion I find interesting. I was confused about the mentioning that the case
distinction k=0,1,2 does not cover "all cases" - all cases of what? And why is
this distinction sufficient for the task at hand?

Most of the rest of the paper felt hard to read, because it read like a
chronological enumeration of all steps. I think that it would be much more
interesting if the paper highlights the interesting aspects of the verification.

Related work:

In reference [13], the order of authors is different than in the original
publication. I think that you might want to discuss [13] in the Related work
section, since it seems quite close. I was wondering whether you consider the
approach in [13] as synthesis or verification, because to me, it seemed a
mix/neither.

Clarification in intro:

The intro states that this is, to the authors knowledge, the first use of
Verifiable Software Toolchain (VST) in software verification of an implementation
of an assymmetric cryptographic primitives. Does this mean that VST has been
used for symmetric cryptographic primitives before? From the text, I wasn't sure
whether this is careful quantification or not.

\subheading{Our answer}

Clarification in intro: VST has already been used to verify symmetric Cryptographic
primities, as mentioned later by \cite{Beringer2015VerifiedCA} and \cite{2015-Appel}.

\subsection{S\&P 2020 Review \#582C}

Overall merit: 3. Weak reject - The paper has flaws, but I will not argue against it.


\begin{center}
  \subheading{===== Brief paper summary (2-3 sentences) =====}
\end{center}

This paper presents a formalization of the X25519 (Curve25519) elliptic curve
Diffie-Hellman  (ECDH) construction in Coq, and uses this formalization to
verify the functional correctness of a C implementation of this construction
taken from the TweetNaCl library. Going further than prior work, the authors
also link their Coq specification to a mathematical theory of elliptic curves
in Coq.


\begin{center}
  \subheading{===== Strengths =====}
\end{center}

Proofs of cryptographic algorithms like X25519 tend to mean different things in
different communities: (a) one may prove that X25519 implements a group based on
the mathematical theory of elliptic curves, (b) one may prove that an X25519
implementation meets its mathematical spec, or (c) one may prove that a protocol
that uses X25519 is provably secure, based on some cryptographic assumption on
the ECDH construction. It is rare to find papers that combine more than one of
these proof levels, and this paper does a creditable job of linking (a) with (b),
relying on the Coq framework to soundly glue these proofs together. Although a
similar goal was considered in [12], this paper provides a more satisfying
solution by not leaving any gaps between the formalizations, and by relying on a
smaller trusted computing base.


\begin{center}
  \subheading{===== Weaknesses =====}
\end{center}

X25519 implementations have now been verifying in multiple papers using multiple
verification frameworks, and [18] even uses Coq to verify a large portion of a C
implementation of X25519 fully automatically. So, the main contribution here is
that that the authors verify an existing popular implementation (TweetNaCl)
without making many changes to the source code. Still, TweetNaCl is not necessarily
the most complex (and certainly not the most efficient) implementation of X25519
that has been verified, which makes the novelty of this work hard to justify.


\begin{center}
  \subheading{===== Detailed comments for the author(s) =====}
\end{center}

I like the approach taken in this paper and I think this work has value and
should be published.
However, I am not sure the current presentation of the work works well.

The paper can be essentially divided into two components: the proof of TweetNaCl
using VST, and the proof of equivalence between the Coq spec of RFC7748 and the
mathematical theory of Curve25519.  The details of both are interesting to
formalists but the Coq details are quite hard to parse and distracting when
trying to understand the text. This is a well-known problem for formal
verification papers, and I don't have great advice except to suggest that the
authors separate out the code fragments into figures and let the text focus
on the high-level ideas of the code.

At the end of Section IV, I would have loved to see some high-level lessons on
what the proof taught you, and what part of this proof would be reusable if you
decided to verify Ed25519 or X448 or P-256, for example. This is particularly
important for this work, since prior papers have shown how to share proof effort
across multiple curves.

It also appears that the authors do not verify the constant-time guarantees of
the TweetNaCl code. Could this be done in their framework?

Section V presents the proof of mathematical equivalence. It appears to me that
this proof is completely independent of the TweetNaCl proof. Is that the case?
Is there any lemma from the first part that is needed in the second? More
importantly, could you claim that your proof of equivalence also provides a
justification for the implementations of Fiat Crypto or HACL*, hence adding
value to those other projects in addition to your own?

I was hoping to see where the "clamping" used in X25519 would appear in the
formalization of Section V but was unable to spot it. Does clamping affect the
proofs in any way? Is there a stronger property you could prove because of the
clamping (e.g. membership in the prime-order group)?
Do you already prove such a property?  More generally, what about the proof of
Section V is specific to X25519 and what is generic?
