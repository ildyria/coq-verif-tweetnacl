\section{The complete X25519 code from TweetNaCl}
\label{verified-C-and-diff}

\subheading{Verified C Code} We provide below the code we verified.

\lstinputlisting[linerange={2-5,8-9,266-317,333-380,393-438},language=Ctweetnacl]{../proofs/vst/c/tweetnaclVerifiableC.c}

\subheading{Diff from TweetNaCl} We provide below the diff between the original code of TweetNaCl and the code we verified.

\lstinputlisting[language=diff]{tweetnacl.diff}

As follow, we provide the explanations of the above changes to TweetNaCl's code.

\begin{itemize}
  \item lines 7-8: We tell VST that \TNaCle{long long} are
  aligned on 8 bytes.

  \item lines 16-23: We remove the the undefined behavior as explained in \sref{sec:Conclusion}.

  \item lines 29-31; lines 60-62: VST does not support \TNaCle{for} loops over \TNaCle{i64}, we convert it into an \TNaCle{int}.

  \item lines 39 \& 41: We initialize \TNaCle{m} with \TNaCle{0}.
  This change allows us to prove the functional correctness of \TNaCle{pack25519} without having to deal with an array containing
  a mix of uninitialized and initialized values.
  A hand iteration over the loop trivially shows that no uninitialized values are used.

  \item lines 40 \& 42; lines 70 \& 71: We replace the \TNaCle{FOR} loop by \TNaCle{set25519}. The code is the same once the function is inlined. This small change is purely cosmetic but stays in the spirit of tweetnacl: keeping a small code size while being auditable.

  \item lines 50-52: VST does not allow computation in the argument before a function call. Additionally \texttt{clightgen} does not extract the computation either. We add this small step to allow VST to carry through the proof.

  % \item lines 60-62: VST does not support \TNaCle{for} loops over \TNaCle{i64}, we convert it into an \TNaCle{int}.

  % \item lines 70-71: We replace \TNaCle{FOR(a,16) c[a]=i[a];} by \TNaCle{set25519(c,i);}. The semantic of operation done is the same once \TNaCle{set25519} is inlined. This small change is purely cosmetic but stays in the spirit of tweetnacl: keeping a small code size while being auditable.

  \item lines 79-82: VST does not support \TNaCle{for} loops over \TNaCle{i64}, we convert it into an \TNaCle{int}.\\
  In the function calls of \TNaCle{sel25519}, the specifications requires the use of \TNaCle{int}, the value of \TNaCle{r} being either \TNaCle{0} or \TNaCle{1}, we consider this change safe.

  \item Lines 90-101: The \TNaCle{for} loop does not add any benefits to the code. By removing it we simplify the source and the verification steps as we do not need to deal with pointer arithmetic. Thus, \TNaCle{x} is limited to only 16 \TNaCle{i64}, \ie \TNaCle{gf}.

\end{itemize}
