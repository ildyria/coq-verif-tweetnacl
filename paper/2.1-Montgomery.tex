\subsection{Arithmetic on Montgomery curves}
\label{subsec:arithmetic-montgomery}

\begin{dfn}
  Given a field \K, and $a,b \in \K$ such that $a^2 \neq 4$ and $b \neq 0$,
  $M_{a,b}$ is the Montgomery curve defined over $\K$ with equation
  $$M_{a,b}: by^2 = x^3 + ax^2 + x.$$
\end{dfn}

\begin{dfn}
  For any algebraic extension $\L \supseteq	\K$, we call
  $M_{a,b}(\L)$ the set of $\L$-rational points, defined as
  $$M_{a,b}(\L) = \{\Oinf\} \cup \{(x,y) \in \L \times \L~|~by^2 = x^3 + ax^2 + x\}.$$
  Here, the additional element $\Oinf$ denotes the point at infinity.
\end{dfn}
Details of the formalization can be found in \sref{subsec:ECC-Montgomery}.


For $M_{a,b}$ over a finite field $\F{p}$, the parameter $b$ is known as the ``twisting factor''.
For $b'\in \F{p}\backslash\{0\}$ and $b' \neq b$, the curves $M_{a,b}$ and $M_{a,b'}$
are isomorphic via $(x,y) \mapsto (x, \sqrt{b/b'} \cdot y)$.

\begin{dfn}
  When $b'/b$ is not a square in \F{p}, $M_{a,b'}$ is a \emph{quadratic twist} of $M_{a,b}$, i.e.,
  a curve that is isomorphic over $\F{p^2}$~\cite{cryptoeprint:2017:212}.
\end{dfn}

Points in $M_{a,b}(\K)$ can be equipped with a structure of an abelian group
with the addition operation $+$ and with neutral element the point at infinity $\Oinf$.
For a point $P \in M_{a,b}(\K)$ and a positive integer $n$ we obtain the scalar product
$$n\cdot P = \underbrace{P + \cdots + P}_{n\text{ times}}.$$

In order to efficiently compute the scalar multiplication we use an algorithm
similar to square-and-multiply: the Montgomery ladder where the basic operations
are differential addition and doubling~\cite{MontgomerySpeeding}.

We consider \xcoord-only operations. Throughout the computation,
these $x$-coordinates are kept in projective representation
$(X : Z)$, with $x = X/Z$; the point at infinity is represented as $(1:0)$.
See \sref{subsec:ECC-projective} for more details.
We define the operation:
\begin{align*}
  \texttt{xDBL\&ADD} & : (x_{Q-P}, (X_P:Z_P), (X_Q:Z_Q)) \mapsto              \\
                     & ((X_{2 \cdot P}:Z_{2 \cdot P}), (X_{P + Q}:Z_{P + Q}))
\end{align*}
In the Montgomery ladder, % notice that
% the arguments of \texttt{xADD} and \texttt{xDBL}
the arguments $P$ and $Q$ of \texttt{xDBL\&ADD}
are swapped depending on the value of the $k^{\text{th}}$ bit.
We use a conditional swap \texttt{CSWAP} to change the arguments of the above
function while keeping the same body of the loop. \label{cswap}
Given a pair $(P_0, P_1)$ and a bit $b$, \texttt{CSWAP} returns the pair
$(P_b, P_{1-b})$.

By using the differential addition and doubling operations we define the Montgomery ladder
computing a \xcoord-only scalar multiplication (see \aref{alg:montgomery-ladder}).
\begin{algorithm}
  \caption{Montgomery ladder for scalar mult.}
  \label{alg:montgomery-ladder}
  \begin{algorithmic}
    \REQUIRE{\xcoord $x_P$ of a point $P$, scalar $n$ with $n < 2^m$}
    \ENSURE{\xcoord $x_Q$ of $Q = n \cdot P$}
    \STATE $Q = (X_Q:Z_Q) \leftarrow (1:0)$
    \STATE $R = (X_R:Z_R) \leftarrow (x_P:1)$
    \FOR{$k$ := $m$ down to $1$}
    \STATE $(Q,R) \leftarrow \texttt{CSWAP}((Q,R), k^{\text{th}}\text{ bit of }n)$
    % \STATE $Q \leftarrow \texttt{xDBL}(Q)$
    % \STATE $R \leftarrow \texttt{xADD}(x_P,Q,R)$
    \STATE $(Q,R) \leftarrow \texttt{xDBL\&ADD}(x_P,Q,R)$
    \STATE $(Q,R) \leftarrow \texttt{CSWAP}((Q,R), k^{\text{th}}\text{ bit of }n)$
    \ENDFOR
    \RETURN $X_Q/Z_Q$
  \end{algorithmic}
\end{algorithm}
