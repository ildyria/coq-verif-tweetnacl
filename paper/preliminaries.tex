\section{Preliminaries}
\label{preliminaries}

In this section, we describe X25519 and TweetNaCl implementation.
We then provide a brief description of the formal tools we use in our proofs.

\subsection{The X25519 key exchange}
\label{preliminaries:A}

% \begin{definition}
%   Let $E$ be the elliptic curve defined by $y^2 = x^3 + 486662 x^2 + x$.
% \end{definition}
% \begin{lemma}
%   For any value $x \in \F{p}$, for the elliptic curve $E$ over $\F{p^2}$
%   defined by $y^2 = x^3 + 486662 x^2 + x$, there exist a point $P$ over $E(\F{p^2})$
% \end{lemma}

For any value $x \in \F{p}$, for the elliptic curve $E$ over $\F{p^2}$
defined by $y^2 = x^3 + 486662 x^2 + x$, there exist a point $P$ over $E(\F{p^2})$
such that $x$ is the $x$-coordinate of $P$. Remark that $x$ is also the $x$-coordinate of $-P$.

Given a natural number $n$ and $x$, X25519 returns the $x$-coordinate of the
scalar multiplication of $P$ by $n$. Note that the result would is the same with $-P$.

Using X25519, RFC~7748~\cite{rfc7748} formalized a Diffie–Hellman key-exchange algorithm.
Each party generate a secret random number $S_a$ (respectively $S_b$), and computes $P_a$ (respectively $P_b$),
the resulting $x$-coordinate of the scalar multiplication of the base point
where $x = 9$ and $S_a$ (respectively $S_b$).
The party exchanges $P_a$ and $P_b$ and computes their shared secret with X25519
over $S_a$ and $P_b$ (respectively $S_b$ and $P_a$).

\subsection{X25519 in TweetNaCl}
\label{preliminaries:B}

\subheading{Arithmetic in \Ffield} In X25519, all computations are done over $\F{p}$.
Numbers in that field can be represented with 256 bits.
We represent them in 8-bit limbs (respectively 16-bit limbs),
making use of a base $2^8$ (respectively $2^{16}$).
Consequently, inputs of the X25519 function are seen as arrays of bytes
in a little-endian format.
Computations inside this function makes use of the 16-bit limbs representation.
Those are placed into 64-bits signed container in order to mitigate overflows or underflows.
\begin{lstlisting}[language=Ctweetnacl]
typedef long long i64;
typedef i64 gf[16];
\end{lstlisting}
Notice this does not guaranty a unique representation of each number. i.e.\\
$\exists x,y \in$ \TNaCle{gf} such that
\vspace{-0.25cm}
$$x \neq y\ \ \land\ \ x \equiv y \pmod{2^{255}-19}$$

On the other hand it allows simple definitions of addition (\texttt{A}),
substraction (\texttt{Z}), and school-book multiplication (\texttt{M}).
 % and squaring (\texttt{S}).
\begin{lstlisting}[language=Ctweetnacl]
sv A(gf o,const gf a,const gf b) {
  int i;
  FOR(i,16) o[i]=a[i]+b[i];
}

sv Z(gf o,const gf a,const gf b) {
  int i;
  FOR(i,16) o[i]=a[i]-b[i];
}

sv M(gf o,const gf a,const gf b) {
  i64 i,j,t[31];
  FOR(i,31) t[i]=0;
  FOR(i,16) FOR(j,16) t[i+j]+=a[i]*b[j];
  FOR(i,15) t[i]+=38*t[i+16];
  FOR(i,16) o[i]=t[i];
  car25519(o);
  car25519(o);
}
\end{lstlisting}

To avoid overflows, carries are propagated by the \texttt{car25519} function.
\begin{lstlisting}[language=Ctweetnacl]
sv car25519(gf o)
{
  int i;
  FOR(i,16) {
    o[(i+1)%16]+=(i<15?1:38)*(o[i]>>16);
    o[i]&=0xffff;
  }
}
\end{lstlisting}
% In order to simplify the verification of this function,
% we extract the last step of the loop $i = 15$.
% \begin{lstlisting}[language=Ctweetnacl]
% sv car25519(gf o)
% {
%   int i;
%   i64 c;
%   FOR(i,15) {
%     o[(i+1)]+=o[i]>>16;
%     o[i]&=0xffff;
%   }
%   o[0]+=38*(o[15]>>16);
%   o[15]&=0xffff;
% }
% \end{lstlisting}

Inverse in $\Zfield$ are computed with \texttt{inv25519}.
It takes the exponentiation by $2^{255}-21$ with the Square-and-multiply algorithm.
Fermat's little theorem brings the correctness.
Notice that in this case the inverse of $0$ is defined as $0$.

% It takes advantage of the shape of the number by not doing the multiplications only twice.

% \todo{Ladder algorithm C code}
% \todo{Ladderstep algorithm C code}
The last step of \texttt{crypto\_scalarmult} is the packing of the limbs: an array of 32 bytes.
It first performs 3 carry propagations in order to guarantee
that each 16-bit limbs values are between $0$ and $2^{16}$.
Then computes a modulo reduction by $\p$ using iterative substraction and
conditional swapping. This guarantees a unique representation in $\Zfield$.
After which each 16-bit limbs are splitted into 8-bit limbs.
\begin{lstlisting}[language=Ctweetnacl]
sv pack25519(u8 *o,const gf n)
{
  int i,j;
  i64 b;
  gf t,m={0};
  set25519(t,n);
  car25519(t);
  car25519(t);
  car25519(t);
  FOR(j,2) {
    m[0]=t[0]- 0xffed;
    for(i=1;i<15;i++) {
      m[i]=t[i]-0xffff-((m[i-1]>>16)&1);
      m[i-1]&=0xffff;
    }
    m[15]=t[15]-0x7fff-((m[14]>>16)&1);
    m[14]&=0xffff;
    b=1-((m[15]>>16)&1);
    sel25519(t,m,b);
  }
  FOR(i,16) {
    o[2*i]=t[i]&0xff;
    o[2*i+1]=t[i]>>8;
  }
}
\end{lstlisting}

\subheading{The Montgomery ladder}
In order to compute scalar multiplications, X25519 uses the Montgomery
$x$-coordinate double-and-add formulas.
First extract and clamp the value of $n$. Then unpack the value of $p$.
As per RFC~7748~\cite{rfc7748}, set its most significant bit to 0.
Finally compute the Montgomery ladder over the clamped $n$ and $p$,
pack the result into $q$.
\begin{lstlisting}[language=Ctweetnacl]
int crypto_scalarmult(u8 *q,
                      const u8 *n,
                      const u8 *p)
{
  u8 z[32];
  i64 r;
  int i;
  gf x,a,b,c,d,e,f;
  FOR(i,31) z[i]=n[i];
  z[31]=(n[31]&127)|64;
  z[0]&=248;
  unpack25519(x,p);
  FOR(i,16) {
    b[i]=x[i];
    d[i]=a[i]=c[i]=0;
  }
  a[0]=d[0]=1;
  for(i=254;i>=0;--i) {
    r=(z[i>>3]>>(i&7))&1;
    sel25519(a,b,r);
    sel25519(c,d,r);
    A(e,a,c);
    Z(a,a,c);
    A(c,b,d);
    Z(b,b,d);
    S(d,e);
    S(f,a);
    M(a,c,a);
    M(c,b,e);
    A(e,a,c);
    Z(a,a,c);
    S(b,a);
    Z(c,d,f);
    M(a,c,_121665);
    A(a,a,d);
    M(c,c,a);
    M(a,d,f);
    M(d,b,x);
    S(b,e);
    sel25519(a,b,r);
    sel25519(c,d,r);
  }
  inv25519(c,c);
  M(a,a,c);
  pack25519(q,a);
  return 0;
}
\end{lstlisting}

\subsection{Coq and VST}
\label{preliminaries:C}


Coq~\cite{coq-faq} is an interactive theorem prover. It provides an expressive
formal language to write mathematical definitions, algorithms and theorems coupled
with their proofs. As opposed to other systems such as F*~\cite{DBLP:journals/corr/BhargavanDFHPRR17},
Coq does not rely on the trust ofa SMT solver. It uses its type
system to verify the applications of hypothesis, lemmas , theorems~\cite{Howard1995-HOWTFN}.

The Hoare logic is a formal system which allows to reason about programs.
It uses tripples such as
$$\{{\color{doc@lstnumbers}\textbf{Pre}}\}\texttt{~Prog~}\{{\color{doc@lstdirective}\textbf{Post}}\}$$
where ${\color{doc@lstnumbers}\textbf{Pre}}$ and ${\color{doc@lstdirective}\textbf{Post}}$
are assertions and \texttt{Prog} is a piece of Code.
It is read as ``when the precondition $Pre$ is met, executing \texttt{Prog} while yield to postcondition $Post$''.
We use compositional rules to prove the truth value of a Hoare tripple.
For example, here is the rule for sequential composition:
\begin{prooftree}
  \AxiomC{$\{P\}C_1\{Q\}$}
  \AxiomC{$\{Q\}C_2\{R\}$}
  \LeftLabel{Hoare-Seq}
  \BinaryInfC{$\{P\}C_1;C_2\{R\}$}
\end{prooftree}
Separation Logic is an extension of the Hoare logic which allows to reason about
pointers and memory manipulation. This provides the strict condition that considered
memory shares are disjoint, forcing non-aliasing. We discuss further about this limitation in Section~\ref{using-VST}.
The Verified Software Toolchain (VST)~\cite{cao2018vst-floyd} is a framework which uses
the Separation logic to prove the functional correctness of C programs.
Its can be seen as a forward symbolic execution of the program.
Its uses a strongest postcondition approach to prove that a code matches its specification.
