\section{Preliminaries}

\subsection{The X25519 key exchange}


XXX: math definition from Cure25519 paper.

\subsection{X25519 in TweetNaCl}

\subheading{The Montgomery ladder}

\todo{Ladder algorithm C code}

\todo{Ladderstep algorithm C code}

\subheading{Arithmetic in \Ffield}
Given a natural number $n$ and a value $x \in \F{p}$, Curve25519 is a function over a $\F{p}$-restricted
$x$-coordinate computing a scalar multiplication on $E(\F{p^2})$.
As a result of this restriction, all computations are done over $\F{p}$.
Numbers in that field can be represented with 256 bits.
We represent them in 8-bit limbs (respectively 16-bit limbs),
making use of a base $2^8$ (respectively $2^{16}$).
Consequently, inputs of the Curve25519 function are seen as arrays of bytes.
Computations inside this function makes use of the 16-bit limbs representation.
Those are placed into 64-bits signed container in order to mitigate overflows or underflows.
\begin{lstlisting}[language=Ctweetnacl]
typedef long long i64;
typedef i64 gf[16];
\end{lstlisting}
Notice this does not guaranty a unique representation of each number. i.e.\\
$\exists x,y \in$ \TNaCle{gf} such that
\vspace{-0.25cm}
  $$x \neq y\ \ \land\ \ x \equiv y \pmod{2^{255}-19}$$

  On the other hand it allows simple definitions of addition (\texttt{A}),
  substraction (\texttt{Z}), and school-book multiplication (\texttt{M}).
   % and squaring (\texttt{S}).
  \begin{lstlisting}[language=Ctweetnacl]
  sv A(gf o,const gf a,const gf b) {
    int i;
    FOR(i,16) o[i]=a[i]+b[i];
  }

  sv Z(gf o,const gf a,const gf b) {
    int i;
    FOR(i,16) o[i]=a[i]-b[i];
  }

  sv M(gf o,const gf a,const gf b) {
    i64 i,j,t[31];
    FOR(i,31) t[i]=0;
    FOR(i,16) FOR(j,16) t[i+j]+=a[i]*b[j];
    FOR(i,15) t[i]+=38*t[i+16];
    FOR(i,16) o[i]=t[i];
    car25519(o);
    car25519(o);
  }
  \end{lstlisting}

  To avoid overflows, carries are propagated by the \texttt{car25519} function.
  \begin{lstlisting}[language=Ctweetnacl]
  sv car25519(gf o)
  {
    int i;
    FOR(i,16) {
      o[(i+1)%16]+=(i<15?1:38)*(o[i]>>16);
      o[i]&=0xffff;
    }
  }
  \end{lstlisting}
  % In order to simplify the verification of this function,
  % we extract the last step of the loop $i = 15$.
  % \begin{lstlisting}[language=Ctweetnacl]
  % sv car25519(gf o)
  % {
  %   int i;
  %   i64 c;
  %   FOR(i,15) {
  %     o[(i+1)]+=o[i]>>16;
  %     o[i]&=0xffff;
  %   }
  %   o[0]+=38*(o[15]>>16);
  %   o[15]&=0xffff;
  % }
  % \end{lstlisting}

  At the end of the Montgomery ladder, \texttt{inv25519} computes the inverse over \Zfield.
  It uses Fermat's little theorem by the exponentiation to
  $2^{255}-21$ with the Square-and-multiply algorithm.
  % It takes advantage of the shape of the number by not doing the multiplications only twice.

  The last step of \texttt{crypto\_scalarmult} is the packing of the limbs: an array of 32 bytes.
  It first performs 3 carry propagations in order to guarantee
  that each 16-bit limbs values are between $0$ and $2^{16}$.
  Then computes a modulo reduction by $\p$ using iterative substraction and
  conditional swapping. This guarantees a unique representation in $\Zfield$.
  After which each 16-bit limbs are splitted into 8-bit limbs.
  \begin{lstlisting}[language=Ctweetnacl]

  sv pack25519(u8 *o,const gf n)
  {
    int i,j;
    i64 b;
    gf t,m={0};
    set25519(t,n);
    car25519(t);
    car25519(t);
    car25519(t);
    FOR(j,2) {
      m[0]=t[0]- 0xffed;
      for(i=1;i<15;i++) {
        m[i]=t[i]-0xffff-((m[i-1]>>16)&1);
        m[i-1]&=0xffff;
      }
      m[15]=t[15]-0x7fff-((m[14]>>16)&1);
      m[14]&=0xffff;
      b=1-((m[15]>>16)&1);
      sel25519(t,m,b);
    }
    FOR(i,16) {
      o[2*i]=t[i]&0xff;
      o[2*i+1]=t[i]>>8;
    }
  }
  \end{lstlisting}

  The full Montgomery ladder is defined as follow.
  First extract and clamp the value of $n$. Then unpack the value of $p$.
  Compute the Montgomery ladder over the clamped $n$ and $p$, pack the result into $q$.
  \begin{lstlisting}[language=Ctweetnacl]
  int crypto_scalarmult(u8 *q,
                        const u8 *n,
                        const u8 *p)
  {
    u8 z[32];
    i64 r;
    int i;
    gf x,a,b,c,d,e,f;
    FOR(i,31) z[i]=n[i];
    z[31]=(n[31]&127)|64;
    z[0]&=248;
    unpack25519(x,p);
    FOR(i,16) {
      b[i]=x[i];
      d[i]=a[i]=c[i]=0;
    }
    a[0]=d[0]=1;
    for(i=254;i>=0;--i) {
      r=(z[i>>3]>>(i&7))&1;
      sel25519(a,b,r);
      sel25519(c,d,r);
      A(e,a,c);
      Z(a,a,c);
      A(c,b,d);
      Z(b,b,d);
      S(d,e);
      S(f,a);
      M(a,c,a);
      M(c,b,e);
      A(e,a,c);
      Z(a,a,c);
      S(b,a);
      Z(c,d,f);
      M(a,c,_121665);
      A(a,a,d);
      M(c,c,a);
      M(a,d,f);
      M(d,b,x);
      S(b,e);
      sel25519(a,b,r);
      sel25519(c,d,r);
    }
    inv25519(c,c);
    M(a,a,c);
    pack25519(q,a);
    return 0;
  }
  \end{lstlisting}

\subsection{Coq and VST}



Verifying \texttt{crypto\_scalarmult} also implies to verify all the functions
subsequently called: \texttt{unpack25519}; \texttt{A}; \texttt{Z}; \texttt{M};
\texttt{S}; \texttt{car25519}; \texttt{inv25519}; \texttt{set25519}; \texttt{sel25519};
\texttt{pack25519}.

We prove that the implementation of Curve25519 is \textbf{sound} \ie
\begin{itemize}
\item absence of access out-of-bounds of arrays (memory safety).
\item absence of overflows/underflow on the arithmetic.
\end{itemize}
We also prove that TweetNaCl's code is \textbf{correct}:
\begin{itemize}
\item Curve25519 is correctly implemented (we get what we expect).
\item Operations on \texttt{gf} (\texttt{A}, \texttt{Z}, \texttt{M}, \texttt{S})
are equivalent to operations ($+,-,\times,x^2$) in \Zfield.
\item The Montgomery ladder does compute a scalar multiplication between a natural number and a point.
\end{itemize}

In order to prove the soundness and correctness of \texttt{crypto\_scalarmult},
we first create a skeleton of the Montgomery ladder with abstract operations which
can be instanciated over lists, integers, field elements...
A high level specification (over a generic field $\K$) allows use to prove the
correctness of the ladder with respect to the curves theory.
This high specification does not rely on the parameters of Curve25519.
By instanciating $\K$ with $\Zfield$, and the parameters of Curve25519 ($a = 486662, b = 1$),
we define a middle level specification.
Additionally we also provide a low level specification close to the \texttt{C} code
(over lists of $\Z$). We show this specification to be equivalent to the
\textit{semantic version} of C (\texttt{CLight}) with VST.
This low level specification gives us the soundness assurance.
By showing that operations over instances ($\K = \Zfield$, $\Z$, list of $\Z$) are
equivalent we bridge the gap between the low level and the high level specification
with Curve25519 parameters.
As such we prove all specifications to equivalent (Fig.\ref{tk:ProofStructure}).
This garantees us the correctness of the implementation.

\begin{figure}[h]
  \begin{tikzpicture}[textstyle/.style={black, anchor= south west, align=center}]

    \filldraw[draw=orange!10!doc@lstbackground, fill=doc@lstbackground, thick] (0.25,0.5) rectangle (4.5,5.5);
    % node[textstyle, anchor=west, draw=yellow, fill=yellow!20, thick, minimum width=5.5cm,minimum height=5cm] {};

    \draw (4.5,5.5)  node[anchor=north east, inner sep=0pt] (russell) {\includegraphics[width=.03\textwidth]{img/coq_logo.png}};

    \draw (0.5,-1) node [textstyle, anchor=west, draw=black, thick, minimum width=3cm,minimum height=0.5cm] (longlong) {\texttt{long long[16]}};
    \draw (0,-1) node [anchor=east] (longlongdef) {\texttt{C} code};
    \draw (2.25,-0.1) node [anchor=west] (app) {\texttt{clightgen}};


    \draw (0.5,1) node [textstyle, anchor=west, draw=black, thick, minimum width=3cm,minimum height=0.5cm] (clight) {\texttt{tptr tlong}};
    \draw (0,1) node [anchor=east] (clightdef) {\texttt{CLight}};

    \draw [thick, ->] (longlong.north) -- (clight.south);

    \begin{scope}[yshift=1 cm,xshift=0 cm]

      \draw (0.5,2) node [textstyle, anchor=west, draw=black, thick, minimum width=3cm,minimum height=0.5cm] (ll) {\texttt{list} $\Z$};
      \draw (0,2) node [anchor=east] (shn) {Low Level};

      \draw [thick,double, <->, >=implies] (clight.north) -- (ll.south);
      \draw (2.25,1) node[anchor=west, inner sep=0pt] (chain) {\includegraphics[width=.07\textwidth]{img/chain.png}};


      \draw (0.5,3) node [textstyle, anchor=west, draw=black, thick, minimum width=3cm,minimum height=0.5cm] (ml) {$\Zfield$};
      \draw (0,3) node [anchor=east] (shn) {Mid Level};

      \draw[thick,double, <->, >=implies] (ll.north) -- (ml.south);

      \draw (0.5,4) node [textstyle, anchor=west, draw=black, thick, minimum width=3cm,minimum height=0.5cm] (hl) {$\K$};
      \draw (0,4) node [anchor=east] (shn) {High Level};

      \draw[thick,double, <-, >=implies] (ml.north) -- (hl.south);
    \end{scope}
\end{tikzpicture}

  \caption{Structural construction of the proof}
  \label{tk:ProofStructure}
\end{figure}

\subsection{Correctness Specification}

We show the soundness of TweetNaCl by proving the following specification matches a pure Coq function.
This defines the equivalence between the Clight representation and a Coq definition of the ladder.

\begin{CoqVST}
Definition crypto_scalarmult_spec :=
DECLARE _crypto_scalarmult_curve25519_tweet
WITH
  v_q: val, v_n: val, v_p: val, c121665:val,
  sh : share,
  q : list val, n : list Z, p : list Z
(*------------------------------------------*)
PRE [ _q OF (tptr tuchar),
     _n OF (tptr tuchar),
     _p OF (tptr tuchar) ]
PROP (writable_share sh;
      Forall (fun x => 0 <= x < 2^8) p;
      Forall (fun x => 0 <= x < 2^8) n;
      Zlength q = 32; Zlength n = 32;
      Zlength p = 32)
LOCAL(temp _q v_q; temp _n v_n; temp _p v_p;
      gvar __121665 c121665)
SEP  (sh [{ v_q }] <<(uch32)-- q;
      sh [{ v_n }] <<(uch32)-- mVI n;
      sh [{ v_p }] <<(uch32)-- mVI p;
      Ews [{ c121665 }] <<(lg16)-- mVI64 c_121665)
(*------------------------------------------*)
POST [ tint ]
PROP (Forall (fun x => 0 <= x < 2^8) (CSM n p);
      Zlength (CSM n p) = 32)
LOCAL(temp ret_temp (Vint Int.zero))
SEP  (sh [{ v_q }] <<(uch32)-- mVI (CSM n p);
      sh [{ v_n }] <<(uch32)-- mVI n;
      sh [{ v_p }] <<(uch32)-- mVI p;
      Ews [{ c121665 }] <<(lg16)-- mVI64 c_121665
\end{CoqVST}

In this specification we state as preconditions:
\begin{itemize}
  \item[] \VSTe{PRE}: \VSTe{_p OF (tptr tuchar)}\\
  The function \texttt{crypto\_scalarmult} takes as input three pointers to
  arrays of unsigned bytes (\VSTe{tptr tuchar}) \VSTe{_p}, \VSTe{_q} and \VSTe{_n}.
  \item[] \VSTe{LOCAL}: \VSTe{temp _p v_p}\\
  Each pointer represent an address \VSTe{v_p},
  \VSTe{v_q} and \VSTe{v_n}.
  \item[] \VSTe{SEP}: \VSTe{sh [{ v_p $\!\!\}\!\!]\!\!\!$ <<(uch32)-- mVI p}\\
  In the memory share \texttt{sh}, the address \VSTe{v_p} points
  to a list of integer values \VSTe{mVI p}.
  \item[] \VSTe{PROP}: \VSTe{Forall (fun x => 0 <= x < 2^8) p}\\
  In order to consider all the possible inputs, we assumed each
  elements of the list \texttt{p} to be bounded by $0$ included and $2^8$
  excluded.
  \item[] \VSTe{PROP}: \VSTe{Zlength p = 32}\\
  We also assumed that the length of the list \texttt{p} is 32. This defines the
  complete representation of \TNaCle{u8[32]}.
\end{itemize}

As Post-condition we have:
\begin{itemize}
  \item[] \VSTe{POST}: \VSTe{tint}\\
  The function \texttt{crypto\_scalarmult} returns an integer.
  \item[] \VSTe{LOCAL}: \VSTe{temp ret_temp (Vint Int.zero)}\\
  The returned integer has value $0$.
  \item[] \VSTe{SEP}: \VSTe{sh [{ v_q $\!\!\}\!\!]\!\!\!$ <<(uch32)-- mVI (CSM n p)}\\
  In the memory share \texttt{sh}, the address \VSTe{v_q} points
  to a list of integer values \VSTe{mVI (CSM n p)} where \VSTe{CSM n p} is the
  result of the \VSTe{crypto_scalarmult} over \VSTe{n} and \VSTe{p}.
  \item[] \VSTe{PROP}: \VSTe{Forall (fun x => 0 <= x < 2^8) (CSM n p)}\\
  \VSTe{PROP}: \VSTe{Zlength (CSM n p) = 32}\\
  We show that the computation for \VSTe{CSM} fits in  \TNaCle{u8[32]}.
\end{itemize}

This specification shows that \VSTe{crypto_scalarmult} in C computes the same
result as \VSTe{CSM} in Coq provided that inputs are within their respective
bounds.
By converting those array of 32 bytes into their respective little-endian value
we prove the correctness of \VSTe{crypto_scalarmult} (Theorem \ref{CSM-correct})
in Coq (for the sake of simplicity we do not display the conversion in the theorem).
\begin{theorem}
\label{CSM-correct}
For all $n \in \N, n < 2^{255}$ and where the bits 1, 2, 5 248, 249, 250
are cleared and bit 6 is set, for all $P \in E(\F{p^2})$,
for all $p \in \F{p}$ such that $P.x = p$,
there exists $Q \in E(\F{p^2})$ such that $Q = nP$ where $Q.x = q$ and $q$ = \VSTe{CSM} $n$ $p$.
\end{theorem}
A more complete description in Coq of Theorem \ref{CSM-correct} with the associated conversions
is as follow:
\begin{lstlisting}[language=Coq]
Theorem Crypto_Scalarmult_Correct:
  forall (n p:list Z) (P:mc curve25519_Fp2_mcuType),
  Zlength n = 32 ->
  Zlength p = 32 ->
  Forall (fun x => 0 <= x /\ x < 2^8) n ->
  Forall (fun x => 0 <= x /\ x < 2^8) p ->
  Fp2_x (ZUnpack25519 (ZofList 8 p)) = P#x0 ->
  ZofList 8 (Crypto_Scalarmult n p) =
    (P *+ (Z.to_nat (Zclamp (ZofList 8 n)))) _x0.
\end{lstlisting}
% Its proof is explained in the next section.
