\section{Organization of the proof files}
\label{appendix:proof-folders}

\subheading{Requirements.}
Our proofs requires the use of \emph{Coq 8.8.2} for the proofs and
\emph{Opam 2.0} to manage the dependencies. We are aware that there exists more
recent versions of Coq; VST; CompCert etc. however to avoid dealing with backward
breaking compatibility we decided to freeze our dependencies.

\subheading{Associated files.}
The repository containing the proof is composed of two folders \textbf{\texttt{packages}}
and \textbf{\texttt{proofs}}.
It aims to be used at the same time as an \emph{opam} repository to manage
the dependencies of the proof and to provide the code.

The actual proofs can be found in the \textbf{\texttt{proofs}} folder in which
the reader will find the directories \textbf{\texttt{spec}} and \textbf{\texttt{vst}}.

\subheading{\texttt{packages/}}
This folder provides all the required Coq dependencies: ssreflect (1.7), 
VST (2.0), CompCert (3.2), the elliptic curves library by Bartzia \& Strub,
and the theorem of quadratic reciprocity.

\subheading{\texttt{proofs/spec/}}
In this folder the reader will find multiple levels of implementation of X25519.
\begin{itemize}
  \item \textbf{\texttt{Libs/}} contains basic libraries and tools to help
        reason with lists and decidable procedures.
  \item \textbf{\texttt{ListsOp/}} defines operators on list such as
        \Coqe{ZofList} and related lemmas using \eg \VSTe{Forall}.
  \item \textbf{\texttt{Gen/}} defines a generic Montgomery ladder which can be
        instantiated with different operations. This ladder is the stub for the
        following implementations.
  \item \textbf{\texttt{High/}} contains the theory of Montgomery curves,
        twists, quadratic extensions and ladder.
        It also proves the correctness of the ladder over $\F{\p}$.
  \item \textbf{\texttt{Mid/}} provides a list-based implementation of the
        basic operations \TNaCle{A}, \TNaCle{Z}, \TNaCle{M} \ldots~and the ladder. It
        makes the link with the theory of Montgomery curves.
  \item \textbf{\texttt{Low/}} provides a second list-based implementation of
        the basic operations \TNaCle{A}, \TNaCle{Z}, \TNaCle{M} \ldots~and the ladder.
        Those functions are proven to provide the same results as the ones in
        \texttt{Mid/}, however their implementation are closer to \texttt{C} in order
        facilitate the proof of equivalence with TweetNaCl code.
  \item \textbf{\texttt{rfc/}} provides our rfc formalization.
        It uses integers for the basic operations \TNaCle{A}, \TNaCle{Z}, \TNaCle{M}
        \ldots and the ladder. It specifies the decoding/encoding of/to byte
        arrays (seen as list of integers) as in RFC~7748.
\end{itemize}

\subheading{\texttt{proofs/vst/}}
Here the reader will find four folders.
\begin{itemize}
  \item \textbf{\texttt{c}} contains the C Verifiable implementation of TweetNaCl.
        \texttt{clightgen} will generate the appropriate translation into Clight.
  \item \textbf{\texttt{init}} contains basic lemmas and memory manipulation
        shortcuts to handle the aliasing cases.
  \item \textbf{\texttt{spec}} defines as Hoare triple the specification of the
        functions used in \TNaCle{crypto_scalarmult}.
  \item \textbf{\texttt{proofs}} contains the proofs of the above Hoare triples
        and thus the proof that TweetNaCl code is sound and correct.
\end{itemize}
