\section{Proving equivalence of X25519 in C and Coq}
\label{C-Coq}

In this section we describe techniques used to prove the equivalence between the
Clight description of TweetNaCl and Coq functions producing similar behaviors.

\subsection{Number Representation and C Implementation}

As described in Section \ref{sec:impl}, numbers in \TNaCle{gf} are represented
in base $2^{16}$ and we use a direct mapping to represent that array as a list
integers in Coq. However in order to show the correctness of the basic operations,
we need to convert this number as a full integer.
\begin{definition}
Let \Coqe{ZofList} : $\Z \rightarrow \texttt{list} \Z \rightarrow \Z$, a parametrized map by $n$ betwen a list $l$ and its
it's little endian representation with a base $2^n$.
\end{definition}
We define it in Coq as:
\begin{lstlisting}[language=Coq]
Fixpoint ZofList {n:Z} (a:list Z) : Z :=
  match a with
  | [] => 0
  | h :: q => h + (pow 2 n) * ZofList q
  end.
\end{lstlisting}
We define a notation where $n$ is $16$.
\begin{lstlisting}[language=Coq]
Notation "Z16.lst A" := (ZofList 16 A).
\end{lstlisting}
We also define a notation to do the modulo, projecting any numbers in $\Zfield$.
\begin{lstlisting}[language=Coq]
Notation "A :GF" := (A mod (2^255-19)).
\end{lstlisting}
Remark that this representation is different from \Coqe{Zmodp}.
However the equivalence between operations over $\Zfield$ and $\F{p}$ is easily proven.

Using these two definitions, we proved intermediates lemmas such as the correctness of the
multiplication \Coqe{M} where \Coqe{M} replicate the computations and steps done in C.
\begin{lemma}
For all list of integers \texttt{a} and \texttt{b} of length 16 representing
$A$ and $B$ in $\Zfield$, the number represented in $\Zfield$ by the list \Coqe{(M a b)}
is equal to $A \times B \bmod \p$.
\end{lemma}
And seen in Coq as follows:
\begin{Coq}
Lemma mult_GF_Zlength :
  forall (a:list Z) (b:list Z),
  Zlength a = 16 ->
  Zlength b = 16 ->
   (Z16.lst (M a b)) :GF =
   (Z16.lst a * Z16.lst b) :GF.
\end{Coq}

\subsection{Inversions in \Zfield}

In a similar fashion we define a Coq version of the inversion mimicking
the behavior of \TNaCle{inv25519} over \Coqe{list Z}.
\begin{lstlisting}[language=Ctweetnacl]
sv inv25519(gf o,const gf a)
{
  gf c;
  int i;
  set25519(c,a);
  for(i=253;i>=0;i--) {
    S(c,c);
    if(i!=2 && i!=4) M(c,c,a);
  }
  FOR(i,16) o[i]=c[i];
}
\end{lstlisting}
We specify this with 2 functions: a recursive \Coqe{pow_fn_rev} to to simulate the for loop and a simple
\Coqe{step_pow} containing the body. Note the off by one for the loop.
\begin{lstlisting}[language=Coq]
Definition step_pow (a:Z) (c g:list Z) : list Z :=
  let c := Sq c in
    if a <>? 1 && a <>? 3
    then M c g
    else c.

Function pow_fn_rev (a:Z) (b:Z) (c g: list Z)
  {measure Z.to_nat a} : (list Z) :=
  if a <=? 0
    then c
    else
      let prev := pow_fn_rev (a - 1) b c g in
        step_pow (b - 1 - a) prev g.
\end{lstlisting}

This \Coqe{Function} requires a proof of termination. It is done by proving the
Well-foundness of the decreasing argument: \Coqe{measure Z.to_nat a}. Calling
\Coqe{pow_fn_rev} 254 times allows us to reproduce the same behavior as the \texttt{Clight} definition.
\begin{lstlisting}[language=Coq]
Definition Inv25519 (x:list Z) : list Z :=
  pow_fn_rev 254 254 x x.
\end{lstlisting}
Similarily we define the same function over $\Z$.
\begin{lstlisting}[language=Coq]
Definition step_pow_Z (a:Z) (c:Z) (g:Z) : Z :=
  let c := c * c in
  if a <>? 1 && a <>? 3
    then c * g
    else c.

Function pow_fn_rev_Z (a:Z) (b:Z) (c:Z) (g: Z)
  {measure Z.to_nat a} : Z :=
  if (a <=? 0)
    then c
    else
      let prev := pow_fn_rev_Z (a - 1) b c g in
        step_pow_Z (b - 1 - a) prev g.

Definition Inv25519_Z (x:Z) : Z :=
  pow_fn_rev_Z 254 254 x x.
\end{lstlisting}
And prove their equivalence in $\Zfield$.
\begin{lstlisting}[language=Coq]
Lemma Inv25519_Z_GF : forall (g:list Z),
  length g = 16 ->
  (Z16.lst (Inv25519 g)) :GF =
  (Inv25519_Z (Z16.lst g)) :GF.
\end{lstlisting}
In TweetNaCl, \TNaCle{inv25519} computes an inverse in $\Zfield$. It uses the
Fermat's little theorem by doing an exponentiation to $2^{255}-21$.
This is done by applying a square-and-multiply algorithm. The binary representation
of $p-2$ implies to always do a multiplications aside for bit 2 and 4, thus the if case.
To prove the correctness of the result we can use multiple strategies such as:
\begin{itemize}
  \item Proving it is special case of square-and-multiply algorithm applied to
  a specific number and then show that this number is indeed $2^{255}-21$.
  \item Unrolling the for loop step-by-step and applying the equalities
  $x^a \times x^b = x^{(a+b)}$ and $(x^a)^2 = x^{(2 \times a)}$. We prove that
  the resulting exponent is $2^{255}-21$.
\end{itemize}
We use the second method for the benefits of simplicity. However it requires to
apply the unrolling and exponentiation formulas 255 times. This can be automated
in Coq with tacticals such as \Coqe{repeat}, but it generates a proof object which
will take a long time to verify.

\subsection{Speeding up with Reflections}

In order to speed up the verification, we use a technique called reflection.
It provides us with flexibility such as we don't need to know the number of
times nor the order in which the lemmas needs to be applied (chapter 15 in \cite{CpdtJFR}).

The idea is to \textit{reflect} the goal into a decidable environment.
We show that for a property $P$, we can define a decidable boolean property
$P_{bool}$ such that if $P_{bool}$ is \Coqe{true} then $P$ holds.
$$reify\_P : P_{bool} = true \implies P$$
By applying $reify\_P$ on $P$ our goal become $P_{bool} = true$.
We then compute the result of $P_{bool}$. If the decision goes well we are
left with the tautology $true = true$.

To prove that the \Coqe{Inv25519_Z} is computing $x^{2^{255}-21}$,
we define a Domain Specific Language.
\begin{definition}
Let \Coqe{expr_inv} denote an expression which is either a term;
a multiplication of expressions; a squaring of an expression or a power of an expression.
And Let \Coqe{formula_inv} denote an equality between two expressions.
\end{definition}
\begin{lstlisting}[language=Coq]
Inductive expr_inv :=
  | R_inv : expr_inv
  | M_inv : expr_inv -> expr_inv -> expr_inv
  | S_inv : expr_inv -> expr_inv
  | P_inv : expr_inv -> positive -> expr_inv.

Inductive formula_inv :=
  | Eq_inv : expr_inv -> expr_inv -> formula_inv.
\end{lstlisting}
The denote functions are defined as follows:
\begin{lstlisting}[language=Coq]
Fixpoint e_inv_denote (m:expr_inv) : Z :=
  match m with
  | R_inv     =>
    term_denote
  | M_inv x y =>
    (e_inv_denote x) * (e_inv_denote y)
  | S_inv x =>
    (e_inv_denote x) * (e_inv_denote x)
  | P_inv x p =>
    pow (e_inv_denote x) (Z.pos p)
  end.

Definition f_inv_denote (t : formula_inv) : Prop :=
  match t with
  | Eq_inv x y => e_inv_denote x = e_inv_denote y
  end.
\end{lstlisting}
All denote functions also take as an argument the environment containing the variables.
We do not show it here for the sake of readability.
Given that an environment, \Coqe{term_denote} returns the appropriate variable.
With such Domain Specific Language we have the equality between:
\begin{lstlisting}[backgroundcolor=\color{white}]
f_inv_denote
 (Eq_inv (M_inv R_inv (S_inv R_inv))
         (P_inv R_inv 3))
  = (x * x^2 = x^3)
\end{lstlisting}
On the right side, \Coqe{(x * x^2 = x^3)} depends on $x$. On the left side,
\texttt{(Eq\_inv (M\_inv R\_inv (S\_inv R\_inv)) (P\_inv R\_inv 3))} does not depend on $x$.
This allows us to use computations in our decision precedure.

We define \Coqe{step_inv} and \Coqe{pow_inv} to mirror the behavior of
\Coqe{step_pow_Z} and respectively \Coqe{pow_fn_rev_Z} over our DSL and
we prove their equality.
\begin{lstlisting}[language=Coq]
Lemma step_inv_step_pow_eq :
  forall (a:Z) (c:expr_inv) (g:expr_inv),
  e_inv_denote (step_inv a c g) =
  step_pow_Z a (e_inv_denote c) (e_inv_denote g).

Lemma pow_inv_pow_fn_rev_eq :
  forall (a:Z) (b:Z) (c:expr_inv) (g:expr_inv),
  e_inv_denote (pow_inv a b c g) =
  pow_fn_rev_Z a b (e_inv_denote c) (e_inv_denote g).
\end{lstlisting}
We then derive the following lemma.
\begin{lemma}
\label{reify}
With an appropriate choice of variables,
\Coqe{pow_inv} denotes \Coqe{Inv25519_Z}.
\end{lemma}

In order to prove formulas in \Coqe{formula_inv},
we have the following a decidable procedure.
We define \Coqe{pow_expr_inv}, a function which returns the power of an expression.
We then compare the two values and decide over their equality.
\begin{Coq}
Fixpoint pow_expr_inv (t:expr_inv) : Z :=
  match t with
  | R_inv   => 1
  (* power of a term is 1. *)
  | M_inv x y =>
    (pow_expr_inv x) + (pow_expr_inv y)
  (* power of a multiplication is
     the sum of the exponents. *)
  | S_inv x =>
    2 * (pow_expr_inv x)
  (* power of a squaring is the double
     of the exponent. *)
  | P_inv x p =>
    (Z.pos p) * (pow_expr_inv x)
  (* power of a power is the multiplication
     of the exponents. *)
  end.

Definition decide_e_inv (l1 l2:expr_inv) : bool :=
  (pow_expr_inv l1) ==? (pow_expr_inv l2).

Definition decide_f_inv (f:formula_inv) : bool :=
  match f with
  | Eq_inv x y => decide_e_inv x y
  end.
\end{Coq}
We prove our decision procedure correct.
\begin{lemma}
\label{decide}
For all formulas $f$, if the decision over $f$ returns \Coqe{true},
then the denoted equality by $f$ is true.
\end{lemma}
Which is formalized as:
\begin{Coq}
Lemma decide_formula_inv_impl :
  forall (f:formula_inv),
  decide_f_inv f = true ->
  f_inv_denote f.
\end{Coq}
By reification to over DSL (lemma \ref{reify}) and by applying our decision (lemma \ref{decide}).
we proved the following theorem.
\begin{theorem}
\Coqe{Inv25519_Z} computes an inverse in \Zfield.
\end{theorem}
\begin{Coq}
Theorem Inv25519_Z_correct :
  forall (x:Z),
  Inv25519_Z x = pow x (2^255-21).
\end{Coq}

From \Coqe{Inv25519_Z_correct} and \Coqe{Inv25519_Z_GF}, we conclude the
functionnal correctness of the inversion over \Zfield.
\begin{corollary}
\Coqe{Inv25519} computes an inverse in \Zfield.
\end{corollary}
\begin{Coq}
Corollary Inv25519_Zpow_GF :
  forall (g:list Z),
  length g = 16 ->
  Z16.lst (Inv25519 g) :GF  =
  (pow (Z16.lst g) (2^255-21)) :GF.
\end{Coq}

\subsection{Packing and other Applications of Reflection}

We prove the functional correctness of \Coqe{Inv25519} with reflections.
This technique can also be used where proofs requires some computing or a small and
finite domain of variable to test e.g. for all $i$ such that $0 \le i < 16$.
Using reflection we prove that we can split the for loop in \TNaCle{pack25519} into two parts.
\begin{lstlisting}[language=Ctweetnacl]
for(i=1;i<15;i++) {
  m[i]=t[i]-0xffff-((m[i-1]>>16)&1);
  m[i-1]&=0xffff;
}
\end{lstlisting}
The first loop is computing the substraction while the second is applying the carrying.
\begin{lstlisting}[language=Ctweetnacl]
for(i=1;i<15;i++) {
  m[i]=t[i]-0xffff
}
for(i=1;i<15;i++) {
  m[i]=m[i]-((m[i-1]>>16)&1);
  m[i-1]&=0xffff;
}
\end{lstlisting}
This loop separation allows simpler proofs. The first loop is seen as the substraction of a number in \Zfield.
We then prove that with the iteration of the second loop, the number represented in \Zfield stays the same.
This leads to the proof that \TNaCle{pack25519} is effectively reducing mod $\p$ and returning a number in base $2^8$.

\begin{Coq}
Lemma Pack25519_mod_25519 :
forall (l:list Z),
Zlength l = 16 ->
Forall (fun x => -2^62 < x < 2^62) l ->
ZofList 8 (Pack25519 l) = (Z16.lst l) mod (2^255-19).
\end{Coq}
