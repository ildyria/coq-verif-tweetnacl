XXX: NaCl
XXX: TweetNaCl (find real-world use of TweetNaCl?)

XXX: 
``TweetNaCl is the
first cryptographic library that allows correct functionality to be verified
by auditors with reasonable effort''

XXX: One core component of TweetNaCl (and NaCl) is X25519, mention use
of X25519 in the wild.



TweetNaCl\cite{BGJ+15} is a compact reimplementation of the
NaCl\cite{BLS12} high security cryptographic library.
It does not aim for high speed application and has been optimized for source
code compactness (100 tweets). It maintains some degree of readability in order
to be easily auditable.

This library makes use of Curve25519\cite{Ber06}, a function over a \F{p}-restricted
$x$-coordinate computing a scalar multiplication on $E(\F{p^2})$, where $p$ is
the prime number $\p$ and $E$ is the elliptic curve $y^2 = x^3 + 486662 x^2 + x$.
This function is used by a wide variety of applications\cite{this-that-use-curve25519}
to establish a shared secret over an insecure channel.

Implementing cryptographic primitives without any bugs is difficult.
While tests provides with code coverage, they still don't cover 100\% of the
possible input values. In order to get formal guaranties a software meets its
specifications, two methodologies exist.

The first one is by synthesizing a formal specification and generating machine
code by refinment in order to get a software correct by construction.
This approach is being used in e.g. the B-method\cite{Abrial:1996:BAP:236705},
F* \cite{DBLP:journals/corr/BhargavanDFHPRR17}, or with Coq \cite{CpdtJFR}.

However this first method cannot be applied to an existing piece of software.
In such case we need to write the specifications and then verify the correctness
of the implementation.

\subheading{Contribution of this paper}

\todo{Proof that TweetNaCl's X25519 code correctly implements math definition from 25519 paper}

\todo{State additional contributions, e.g., extension of EC framework by Bartiza and Strub.}

\subheading{Our Formal Approach.}
Verifying an existing cryptographic library, we use the second approach.
Our method can be seen as a static analysis over the input values coupled
with the formal proof that the code of the algorithm matches its specification.

We use Coq\cite{coq-faq}, a formal system that allows us to machine-check our proofs.
A famous example of its use is the proof of the Four Color Theorem \cite{gonthier2008formal}.
The CompCert, a C compiler\cite{Leroy-backend} proven correct and sound is being build on top of it.
To prove its correctness, CompCert uses multiple intermediate languages. The first step of CompCert is done by the parser \textit{clightgen}.
It takes as input C code and generates its Clight\cite{Blazy-Leroy-Clight-09} translation.

Using this intermediate representation Clight, we use the Verifiable Software Toolchain
(VST)\cite{2012-Appel}, a framework which uses Separation logic\cite{1969-Hoare,Reynolds02separationlogic}
and shows that the semantics of the program satisfies a functionnal specification in Coq.
VST steps through each line of Clight using a strongest post-condition strategy.
We write a specification of the crypto scalar multiplication of TweetNaCl and using
VST we prove that the code matches our definitions.

Bartzia and Strub wrote a formal library for elliptic curves\cite{DBLP:conf/itp/BartziaS14}.
We extend it to support Montgomery curves. With this formalization, we prove the
correctness of a generic Montgomery ladder and show that our specification is an instance of it.

\subheading{Related work.}

\todo{Separate verification of existing code from generating proof-carrying code.}

Similar approaches have been used to prove the correctness of OpenSSL HMAC~\cite{Beringer2015VerifiedCA} 
and SHA-256~\cite{2015-Appel}. Compared to their work
our approaches makes a complete link between the C implementation and the formal
mathematical definition of the group theory behind elliptic curves.

Using the synthesis approach, Zinzindohou{\'{e}} et al. wrote an verified extensible
library of elliptic curves\cite{Zinzindohoue2016AVE}. This served as ground work for the
cryptographic library HACL*\cite{zinzindohoue2017hacl} used in the NSS suite from Mozilla.

Coq does not only allows verification but also synthesis.
Using correct-by-construction finite field arithmetic\cite{Philipoom2018CorrectbyconstructionFF}
one can synthesize\cite{Erbsen2016SystematicSO} certified elliptic curve
implementations\cite{Erbsen2017CraftingCE}. These implementation are now being used in BoringSSL\cite{fiat-crypto}.

Curve25519 implementations has already been under the scrutiny \cite{Chen2014VerifyingCS}.
However in this paper we provide a proofs of correctness and soundness of a C program up to
the theory of elliptic curves.

\todo{Add 1-2 sentences about how this compares? Different limitations etc.}

\subheading{Reproducing the proofs.}
To maximize reusability of our results we placed the code of our formal proofs
presented in this paper into the public domain. They are available at XXXXXXX
with instructions of how to compile and verify our proofs.

\subheading{Organization of this paper.}
Section~\ref{sec2-implem} gives the necessary background on Curve25519
implementation in TweetNaCl and provides the specifications we later prove correct.
Section~\ref{sec3-maths} describes our extension of the formal library by Bartzia and Strub.
Section~\ref{sec4-refl} makes the link between the mathematical model and the C implementation.
In this section we also describe some of the techniques we used to speed up some of the proofs.
Section~\ref{sec5-vst} provides with attention points a VST user should be careful
of in order to avoid unnecessary work.


% Five years ago:
% \url{https://www.imperialviolet.org/2014/09/11/moveprovers.html}
% \url{https://www.imperialviolet.org/2014/09/07/provers.html}

% \section{Related Works}
%
% \begin{itemize}
%   \item HACL*
%   \item Proving SHA-256 and HMAC
%   \item \url{http://www.iis.sinica.edu.tw/~bywang/papers/ccs17.pdf}
%   \item \url{http://www.iis.sinica.edu.tw/~bywang/papers/ccs14.pdf}
%   \item \url{https://cryptojedi.org/crypto/#gfverif}
%   \item \url{https://cryptojedi.org/crypto/#verify25519}
%   \item Fiat Crypto : synthesis
% \end{itemize}
%
% Add comparison with Fiat-crypto

