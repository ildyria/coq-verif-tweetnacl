NaCl~\cite{BLS12} is an easy-to-use high-security high-speed software library
for network communication, encryption, decryption, signatures, etc.
TweetNaCl~\cite{BGJ+15} is its compact reimplementation.
It does not aim for high speed application and has been optimized for source
code compactness (100 tweets). It maintains some degree of readability in order
to be easily auditable. For example TweetNaCl is being used by ZeroMQ~\cite{zmq}
messaging queue system to provide portability to its users.
% ``TweetNaCl is the first cryptographic library that allows correct functionality
% to be verified by auditors with reasonable effort''~\cite{BGJ+15}

% XXX: TweetNaCl (find real-world use of TweetNaCl?)
% Mega.nz uses tweetnacl-js (as JS port of tweetnacl) for their webclient https://mega.nz/
% Keybase client: https://github.com/keybase/node-nacl/blob/master/lib/tweetnacl.js

One core component of TweetNaCl (and NaCl) is the key exchange protocol X25519~\cite{rfc7748}.
This protocol is being used by a wide variety of applications~\cite{this-that-use-curve25519}
such as SSH, Signal Protocol, Tor, Zcash, TLS to establish a shared secret over
an insecure channel.

This library makes use of Curve25519~\cite{Ber06}, a function over a \F{p}-restricted
$x$-coordinate computing a scalar multiplication on $E(\F{p^2})$, where $p$ is
the prime number $\p$ and $E$ is the elliptic curve $y^2 = x^3 + 486662 x^2 + x$.

Originally, the name ``Curve25519'' referred to this key exchange protocol,
but Bernstein suggested to rename the scheme to X25519 and to use the name
Curve25519 for the underlying elliptic curve~\cite{Ber14}.
We make use of this notation in this paper.

\subheading{Contribution of this paper}
We provide a mechanized formal proof of the correctness of X25519
implementation in TweetNaCl. This is done by first extending the formal library for
elliptic curves~\cite{DBLP:conf/itp/BartziaS14} of Bartzia and Strub wrote to
support Curve25519. Then we prove that the code correctly implements the definitions
from the original paper by Bernstein~\cite{Ber14}.

% Implementing cryptographic primitives without any bugs is difficult.
% While tests provides with code coverage, they still don't cover 100\% of the
% possible input values.
In order to get formal guaranties a software meets its
specifications, two methodologies exist.

The first one is by synthesizing a formal specification and generating machine
code by refinment in order to get a software correct by construction.
This approach is being used in e.g. the B-method~\cite{Abrial:1996:BAP:236705},
F*~\cite{DBLP:journals/corr/BhargavanDFHPRR17}, or with Coq~\cite{CpdtJFR}.

However this first method cannot be applied to an existing piece of software.
In such case we need to write the specifications and then verify the correctness
of the implementation.

\subheading{Our Formal Approach.}
Verifying an existing cryptographic library, we use the second approach.
Our method can be seen as a static analysis over the input values coupled
with the formal proof that the code of the algorithm matches its specification.

We use Coq~\cite{coq-faq}, a formal system that allows us to machine-check our proofs.
Some famous examples of its use are the proof of the Four Color Theorem~\cite{gonthier2008formal}; or
CompCert, a C~compiler~\cite{Leroy-backend} which proven correct and sound by being build on top of it.
In its proof CompCert uses multiple intermediate languages and show equivalence between them.
The first step of the compilation by CompCert is done by the parser \textit{clightgen}.
It takes as input C code and generates its Clight~\cite{Blazy-Leroy-Clight-09} translation.

Using this intermediate representation Clight, we use the Verifiable Software Toolchain
(VST)~\cite{2012-Appel}, a framework which uses Separation logic~\cite{1969-Hoare,Reynolds02separationlogic}
and shows that the semantics of the program satisfies a functionnal specification in Coq.
VST steps through each line of Clight using a strongest post-condition strategy.
We write a specification of the crypto scalar multiplication of TweetNaCl and using
VST we prove that the code matches our definitions.

Bartzia and Strub wrote a formal library for elliptic curves~\cite{DBLP:conf/itp/BartziaS14}.
We extend it to support Montgomery curves. With this formalization, we prove the
correctness of a generic Montgomery ladder and show that our specification is an instance of it.

\subheading{Related work.}

\todo{Separate verification of existing code from generating proof-carrying code.}

Similar approaches have been used to prove the correctness of OpenSSL HMAC~\cite{Beringer2015VerifiedCA}
and SHA-256~\cite{2015-Appel}. Compared to their work
our approaches makes a complete link between the C implementation and the formal
mathematical definition of the group theory behind elliptic curves.

Using the synthesis approach, Zinzindohou{\'{e}} et al. wrote an verified extensible
library of elliptic curves~\cite{Zinzindohoue2016AVE}. This served as ground work for the
cryptographic library HACL*~\cite{zinzindohoue2017hacl} used in the NSS suite from Mozilla.

Coq does not only allows verification but also synthesis.
Using correct-by-construction finite field arithmetic~\cite{Philipoom2018CorrectbyconstructionFF}
one can synthesize~\cite{Erbsen2016SystematicSO} certified elliptic curve
implementations~\cite{Erbsen2017CraftingCE}. These implementation are now being used in BoringSSL~\cite{fiat-crypto}.

Curve25519 implementations has already been under the scrutiny~\cite{Chen2014VerifyingCS}.
However in this paper we provide a proofs of correctness and soundness of a C program up to
the theory of elliptic curves.

Synthesis approaches forces the user to depend on generated code which may not
be optimal (speed, compactness...), they require to tweak the compiler in order
to achieve the desired properties. On the other hand a carefully handcrafted
optimized code will force the verifier to consider all the possible special cases
and write a complex low level specification in order to prove the correctness of
the algorithm.

\subheading{Reproducing the proofs.}
To maximize reusability of our results we placed the code of our formal proofs
presented in this paper into the public domain. They are available at XXXXXXX
with instructions of how to compile and verify our proofs.

\subheading{Organization of this paper.}
Section~\ref{preliminaries} give the necessary background on Curve25519
implementation and a rough understand of how formal verification works.
Section~\ref{C-Coq} provides the specification of X25519 in addition to proofs
techniques used to show the correctness.
Section~\ref{sec3-maths} describes our extension of the formal library by Bartzia
and Strub and the correctness of X25519 implementation with respect to Bernstein
specifications~\cite{Ber14}.
And lastly in Section~\ref{Conclusion} we discuss the trust in this proof.


% Five years ago:
% \url{https://www.imperialviolet.org/2014/09/11/moveprovers.html}
% \url{https://www.imperialviolet.org/2014/09/07/provers.html}

% \section{Related Works}
%
% \begin{itemize}
%   \item HACL*
%   \item Proving SHA-256 and HMAC
%   \item \url{http://www.iis.sinica.edu.tw/~bywang/papers/ccs17.pdf}
%   \item \url{http://www.iis.sinica.edu.tw/~bywang/papers/ccs14.pdf}
%   \item \url{https://cryptojedi.org/crypto/#gfverif}
%   \item \url{https://cryptojedi.org/crypto/#verify25519}
%   \item Fiat Crypto : synthesis
% \end{itemize}
%
% Add comparison with Fiat-crypto
