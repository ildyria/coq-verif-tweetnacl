\section{Conclusion}
\label{sec:Conclusion}

Any formal system relies on a trusted base. In this section we describe our
chain of trust.

\subheading{Trusted Code Base of the proof.}
Our proof relies on a trusted base, i.e. a foundation of definitions that must be
correct. One should not be able to prove a false statement in that system, \eg, by
proving an inconsistency.

In our case we rely on:
\begin{itemize}
  \item \textbf{Calculus of Inductive Construction}. The intuitionistic logic
  used by Coq must be consistent in order to trust the proofs. As an axiom,
  we assumed that the functional extensionality, which is also consistent with that logic.
  $$\forall x, f(x) = g(x) \implies f = g$$
\begin{lstlisting}[language=Coq]
Lemma f_ext: forall (A B:Type),
  forall (f g: A -> B),
  (forall x, f(x) = g(x)) -> f = g.
\end{lstlisting}

  \item \textbf{Verifiable Software Toolchain}. This framework developed at
  Princeton allows a user to prove that a Clight code matches pure Coq
  specification.

  \item \textbf{CompCert}. The formally proven compiler. We trust that the Clight
  model captures correctly the C17 standard.
  Our proof also assumes that the TweetNaCl code will behave as expected if
  compiled under CompCert.
  % We do not provide guarantees for other C compilers such as Clang or GCC.

  \item \textbf{\texttt{clightgen}}. The tool making the translation from \textbf{C} to
  \textbf{Clight}. It is the first step of the compilation.
  VST does not support the direct verification of \texttt{o[i] = a[i] + b[i]}.
  This required us to rewrite the lines into:
\begin{lstlisting}[language=C]
aux1 = a[i];
aux2 = b[i];
o[i] = aux1 + aux2;
\end{lstlisting}
  The trust of the proof relied on the trust of a correct translation from the
  initial version of \emph{TweetNaCl} to \emph{TweetNaclVerifiableC}.
  \texttt{clightgen} comes with \texttt{-normalize} flag which
  factors out function calls and assignments from inside subexpressions.
  The changes required for a C-code to make it Verifiable are now minimal.

  \item Last but not the least, we must trust: the \textbf{Coq kernel} and its
  associated libraries; the \textbf{Ocaml compiler} on which we compiled Coq;
  the \textbf{Ocaml Runtime} and the \textbf{CPU}. Those are common to all proofs
  done with this architecture \cite{2015-Appel,coq-faq}.
\end{itemize}

\subheading{Corrections in TweetNaCl.}
As a result of this verification, we removed superfluous code.
Indeed indexes 17 to 79 of the \TNaCle{i64 x[80]} intermediate variable of
\TNaCle{crypto_scalarmult} were adding unnecessary complexity to the code,
we removed them.

Peter Wu and Jason A. Donenfeld brought to our attention that the original
\TNaCle{car25519} function presented risk of Undefined Behavior if \texttt{c}
is a negative number.
\begin{lstlisting}[language=Ctweetnacl]
c=o[i]>>16;
o[i]-=c<<16; // c < 0 = UB !
\end{lstlisting}
By replacing this statement by a logical \texttt{and} (and proving the correctness)
we solved this problem.
\begin{lstlisting}[language=Ctweetnacl]
o[i]&=0xffff;
\end{lstlisting}

We believe that the type change of the loop index (\TNaCle{int} instead of \TNaCle{i64})
does not impact the trust of our proof.

\subheading{A complete proof.}

We provide a mechanized formal proof of the correctness of the X25519
implementation in TweetNaCl.
We first formalized X25519 from RFC~7748~\cite{rfc7748} in Coq. Then we proved
that TweetNaCl's implementation of X25519 matches our formalization.
In a second step we extended the Coq library for elliptic curves \cite{BartziaS14}
by Bartzia and Strub to support Montgomery curves. Using this extension we
proved that the X25519 from the RFC and its implementation in TweetNaCl matches
the mathematical definitions as given in~\cite[Sec.~2]{Ber06} (\tref{thm:Elliptic-CSM}).
