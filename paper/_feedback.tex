\documentclass{article}

\usepackage{amsmath}
\usepackage{amsfonts}
\usepackage{url}

\renewcommand{\>}{\quad\nobreak$\longrightarrow$\quad}

\def\`#1'{`#1' (\texttt{#1})}
\def\L#1 #2 #3\par{\item[Line #1:] (\textit{#2}) #3\par}
\def\LL#1-#2 #3 #4\par{\item[Lines #1--#2:] (\textit{#3}) #4\par}
\def\D#1 #2 #3\par{\item[Definition #1:] (\textit{#2}) #3\par}
\def\A#1 #2 #3\par{\item[Algorithm #1:] (\textit{#2}) #3\par}
\def\Lem#1 #2 #3\par{\item[Lemma #1:] (\textit{#2}) #3\par}

\newcommand{\F}{\mathbb F}
\newcommand{\K}{\mathbb K}
\newcommand{\N}{\mathbb N}
\newcommand{\Z}{\mathbb Z}

\makeatletter
\def\newlatin#1#2{\def#1{\@latin{#2}}}
\def\@latin#1{\@ifnextchar.{\@latinfinal{#1}}{\@latinmedial{#1}}}
\def\@latinfinal#1{\emph{#1}\@}
\def\@latinmedial#1{\emph{#1.}}
\makeatother
\newlatin\eg{e.g}
\newlatin\etc{etc}
\newlatin\etseq{et seq}
\newlatin\ie{i.e}
\newlatin\vs{vs}

\begin{document}

\begin{tabular}{rl}
url:&\url{https://viguier.nl/tweetverif.pdf}\\
date:&2019--09--30T20:27:26+0200\\
sha256:&\texttt{\small 5a59b0d06357b20bdba25a4c4c101c2da7fcb387f2f254e79d510388ec4270cb}
\end{tabular}

\begin{description}\let\endgraf=\par

\L21    consistency `RFC-7748' \> `RFC~7748'

\L24    grammar `protocol is a an' \> `protocol is an'

\L25    grammar `$x$-coordinate only' \> `$x$-coordinate-only'

\L25    nit \`Diffie-Hellman' \> \`Diffie--Hellman'

\L28    consistency `key exchange' \> `key-exchange'
        \\
        (or otherwise be consistent about hyphen \vs no hyphen)

\LL34-35 style
        `This proof is done in three steps: we first formalize
         RFC~7748 in Coq.' \endgraf
        The post-colon clause should be a separate sentence (and maybe
         separate paragraph) like the other parts; otherwise this looks
         like you're saying all three parts are to formalize RFC~7748
         in Coq.

\L36    style `a second step' \> `the second step,'

\setbox0=\hbox{\verb!\cite{...}, \cite{...}!}
\setbox2=\hbox{\verb!\cite{..., ...}!}
\L40    style
        `logic [8], [9] to show'
        (\texttt{logic \box0\ to show})
        \>
        `logic [8, 9] to show'
        (\texttt{logic \box2\ to show})
        \endgraf (Maybe this is just the IEEE bibliography style?)

\L45    style `a last step' \> `the last step,'

\L47    style `accomplish this step of the proof' \> `do this'

\L49    style
        `Montgomery curves (and in particular Curve25519).'
        \>
        `Montgomery curves, and in particular Curve25519.'

\LL83-84 misc
        This URL goes to an HTML page with a link to the real file.
        Might be helpful if \texttt{curl} and \texttt{wget} worked on
         the URL\@.
        Or maybe this should be a link to a Git repository instead,
         with a tag for the publication?

\item[Lines 105 \etseq:]
        Line numbering stops here for definitions~II.1 and~II.2, but
         applies to definition~II.3.
        Weird!

\D II.1 grammar
        \textit{`Given a field $\K$\textbf{, let} $a,b \in \K$ such that
         $a^2 \ne 4$ and $b \ne 0$, $M_{a,b}$ is the Montgomery curve
         defined over~$\K$ with equation $\dots$.}
        \>
        \textit{`Given a field $\K$\textbf{ and} $a,b \in \K$ such that
         $a^2 \ne 4$ and $b \ne 0$, $M_{a,b}$ is the Montgomery curve
         defined over~$\K$ with equation $\dots$.}

\D II.2 style
        \textit{`we call $M_{a,b}(\mathbb L)$ the set of
          $\mathbb L$-rational points defined as'}
        \>
        \textit{`we call $M_{a,b}(\mathbb L)$ the set of
          $\mathbb L$-rational points\textbf, defined as'}
        \\
        (add comma)

\setbox0=\hbox{\verb!$k^{\mathrm{th}}$!}
\setbox2=\hbox{\verb!$k^{\mathit{th}}$!}
\A 1    consistency
        `$k^{\mathrm{th}}$' \> `$k^{\mathit{th}}$'
        \\
        (or change line~119 the other way)

\L130   grammar
        `there exist a point' \> `there exists a point'

\LL140-142 grammar
        `by setting bit $255$ of $n$ to \texttt 0; setting bit $254$
         to \texttt 1 and setting the lower $3$ bits to \texttt 0.'
        \\\>\\
        `by setting bit $255$ of $n$ to \texttt 0; setting bit $254$
         to \texttt 1\textbf; and setting the lower $3$ bits to
         \texttt 0.'
        \\
        (missing semicolon; alternatively, use commas instead of
         semicolons)

\L209   style
        `Also multiplication (\texttt M) is heavily exploiting the redundancy'
        \>
        `Also multiplication (\texttt M) heavily exploits the redundancy'
        \\
        (or `makes heavy use of the redundancy')

\L222   style
        `the limbs of the result \texttt o are'
         is confusing---in that typeface, it looks like `the limbs of
         the result $\circ$ are' (\ie,
         \texttt{\$\string\circ\$})---but I guess there's no other
         way\dots

\L517   grammar
        `(See IV-B)' \> `(see IV-B)'

\L679   grammar
        `(e.g. all' \> `(e.g., all'

\L682   grammar
        `However we must' \> `However\textbf, we must'

\L683   grammar
        (missing line numbers in the next paragraph)
        \endgraf
        `Assume its recursive call:
         $f : \N \to \mathit{State} \to \mathit{State}$
         which iteratively applies $g$ with decreasing index:'
        \\\>\\
        `Define the recursion
         $f\colon \N \to \mathit{State} \to \mathit{State}$
         which iteratively applies $g$ with decreasing index:'
        \endgraf
        `Then we have :' \> `Then we have:' (extra space)

\LL688-690 style
        `\{0;1;2;3;4\}' \> `\{0,1,2,3,4\}'
        \\
        (unless you mean something other than the standard set
         notation for `the set of integers having the elements $0$,
         $1$, $2$, $3$, and $4$')

\L700   grammar
        `However in order to show' \> `However\textbf, in order to show'

\L713   consistency
        `(\emph{i.e.}\ under \texttt{:GF})'
        \>
        `(i.e., under \texttt{:GF})'
        \\
        (or otherwise be consistent about italic \vs roman face for
         Latin phrases, which are in roman face elsewhere; consider
         defining macros \texttt{\string\ie} \etc---subsequent cases
         not listed here since you can search \& replace)

\Lem IV.1 grammar
        `\textit{\texttt{Low.M} implements correctly the
         multiplication over $\Z_{2^{255}-19}$.}'
        \\\>
        `\textit{\texttt{Low.M} correctly implements
         multiplication over $\Z_{2^{255}-19}$.}'

\L720   grammar
        `And specified in Coq as follows:'
        \>
        `We specify $\Z_{2^{255}-19}$ multiplication in Coq as follows:'

\L729   grammar
        `However for our purpose'
        \>
        `However\textbf, for our purpose'

\L733   grammar
        `if all the values'
        \>
        `If all the values'
        \endgraf
        It is unclear from the text of the lemma whether the
         constraint is inclusive or exclusive, and the Coq code below
         has one way for $-2^{26}$ to $2^{26}$ and another way for
         $-38$ to $2^{16} + 38$.

\L814   style
        `By using Lemma~IV.1' \> `Using Lemma~IV.1'

\LL826-829 style
        `It uses Fermat's little theorem by doing an exponentiation to
         $2^{255} - 21$.
        This is done by applying a square-and-multiply algorithm.
        The binary representation of $p - 2$ implies to always do
         multiplications except for bits $2$ and $4$.'
        \\\>\\
        `It uses Fermat's little theorem by raising its input to the
         power of $2^{255} - 21$ with a square-and-multiply
         algorithm.
        The binary representation of $p - 2$ implies that every step
         does a multiplication except for bits $2$ and $4$.'

\LL830-831 style
        `we can use multiple strategies such as:'
        \>
        `we could use one of several strategies:'
        \\
        (`multiple strategies' suggests \emph{both}, not \emph{either})

\L837   style
        `for the benefits of simplicity'
        \>
        `because it is simpler'

\L838   grammar
        `However it requires to apply'
        \>
        `However, it requires us to apply'

\L840   terminology
        `tacticals' \> `tactics'

\LL870-871 grammar
        `The first loop is computing the subtraction while the second
         is applying the carries.'
        \\\>\\
        `The first loop computes the subtraction, and the second
         applies the carries.'

\LL895-898 style
        `By using each functions \dots, we defined a Coq definition
         \texttt{Crypto\_Scalarmult} mimicking the exact behavior of
         X25519 in TweetNaCl.'
        \\\>\\
        `Using the functions \dots, we have defined
         \texttt{Crypto\_Scalarmult} in Coq to mimic the exact
         behavior of X25519 in TweetNacl.'

\LL900-903 style
        `By proving that each functions \dots are behaving over
         \texttt{list Z} as their equivalent over \texttt Z with'
        \\\>\\
        `By proving that the functions \dots behave over
         \texttt{list Z} as their equivalent over \texttt Z with'

\LL904-905 nit
        `given the same inputs \texttt{Crypto\_Scalarmult}
         applies the same computation as \texttt{RFC}'
        \\\>\\
        `given the same inputs \texttt{Crypto\_Scalarmult}
         performs the same computation as \texttt{RFC}'

\LL915-916 grammar
        `that TweetNaCl's X25519 implementation respect RFC~7748'
        \\\>\\
        `that TweetNaCl's X25519 implementation respect\textbf s RFC~7748'

\D V.1  grammar
        `\textit{Let a field $\K$, using an appropriate choice of
         coordinates, an elliptic curve $E$ is}'
        \>
        `\textit{Fix a field $\K$.  With an appropriate choice of
         coordinates, an elliptic curve $E$ is}'
        \endgraf
        `\textit{(i.e. no cusps}'
        \>
        `\textit{(i.e., no cusps}'
        \endgraf
        `the solutions $(x, y)$ of $E$ augmented by a distinguished point'
        \>
        `the solutions $(x, y)$ of $E$ together with a distinguished point'
        \endgraf
        `distinguished point $\mathcal O$ (called point at infinity)'
        \>
        `distinguished point $\mathcal O$ called the point at infinity'

\D V.2  grammar
        `\textit{Let $a \in \K$\textbf, and $b \in K$ such that}'
        \>
        `\textit{Let $a \in \K$ and $b \in K$ satisfy}'

\LL951-953 dashes
        `-- the type of fields \dots \texttt{E : ecuType} -- a record'
        \>
        `--- the type of fields \dots \texttt{E : ecuType} --- a record'
        \\
        (em dash with \texttt{---}, not en dash with \texttt{--})

\L968   style
        `Points of an elliptic curve'
        \>
        `Points on an elliptic curve'

\LL970-971 grammar
        `The negation of a point $P = (x, y)$ by taking the symmetric
         with respect to the x axis $-P = (x, -y)$.'
        \>
        `The negation of a point $P = (x, y)$ is defined by reflection
         over the $x$ axis: $-P = (x, -y)$.'

\LL972-974 grammar
        `negation of third intersection of the line passing by $P$ and
         $Q$ or'
        \>
        `negation of the third intersection of the line passing
         through $P$ and $Q$, or'

\L977   grammar
        `defined in Coq as follow:'
        \>
        `defined in Coq as follows:'

\L993   unclear
        `And are proven internal to the curve (with coercion):'
        \\
        Not sure what this means.  Maybe you meant `We prove the curve
         is closed under negation and addition', or `We prove negation
         and addition preserve the curve equation', and a note about
         coercion in Coq?

\L1004  tense
        `we defined the parametric type'
        \>
        `we define the parametric type'
        \\
        (or be consistent about present \vs past tense)

\L1006-1008 dashes
        `a \texttt{K : ecuFieldType} -- the type \dots $2$ or $3$ -- and'
        \>
        `a \texttt{K : ecuFieldType} --- the type \dots $2$ or $3$ --- and'
        \\
        (em dash, not en dash)

\L1021  grammar
        `We define the addition on'
        \>
        `We define addition on'

\L1022  grammar
        `however the actual'
        \>
        `however\textbf, the actual'

\L1039  unclear
        as on line~993

\LL1063-1065 style
        `represented with a triple $(X : Y : Z)$.  With the exception
         of $(0 : 0 : 0)$, any points can be projected.'
        \\
        I think it would be clearer to say `represented as a triple
         $(X : Y : Z)$ where $X$, $Y$, and $Z$ are not all zero.'
        In particular, writing the invalid notation $(0 : 0 : 0)$ is a
         little confusing.

\LL1065-1066
        `Scalar multiples are representing the same point'
        \>
        `Scalar multiples represent the same point'

\LL1066-1067 grammar
        `$(X : Y : Z)$ are $(\lambda X : \lambda Y : \lambda Z)$ defining'
        \>
        `$(X : Y : Z)$ and $(\lambda X : \lambda Y : \lambda Z)$ define'

\L1069  grammar
        `on the Euclidean plane, likewise the point $(X, Y)$'
        \>
        `on the Euclidean plane; likewise, the point $(X, Y)$'

\D V.6  style
        This definition looks a bit funny to me.
        Here's how I'd typeset it:
        \endgraf

        \begin{quotation}
          \noindent
          \textbf{Definition~V.6}\;\it
            Define $\chi\colon M_{a,b}(\K) \to \K \cup \{\infty\}$
             and $\chi_0\colon M_{a,b}(\K) \to \K$ by:
            \begin{equation*}\begin{aligned}
              \chi(\mathcal O) &= \infty, & \chi\bigl((x, y)\bigr) &= x; \\
              \chi_0(\mathcal O) &= 0, & \chi_0\bigl((x, y)\bigr) &= x. \\
            \end{aligned}\end{equation*}
        \end{quotation}

        (Using \texttt{\string\begin\string{aligned\string}} inside
         the display saves a tiny bit of vertical space if the
         preceding line is short!)

\L1091 grammar
        `then for any point $P_1$'
        \>
        `Then for any point $P_1$'
        \\
        (and change `,' to `.' at end of preceding display)

\L1098 grammar
        `then for any point $P_1$'
        \>
        `Then for any point $P_1$'
        \\
        (and change `,' to `.' at end of preceding display)

\L1116  style
        `We can remark that'
        \>
        `We remark that'
        \\
        (That said, it looks like you're saying you can \emph{prove}
         that the ladder returns $0$ for $x = 0$, not merely
         \emph{remark} it.)

\L1120  style
        `Also $\mathcal O$ is the neutral element'
        \>
        `As $\mathcal O$ is the neutral element'

\L1121  grammar
        `thus we derive the following lemma'
        \>
        `Thus we derive the following lemma'
        \\
        (or put a semicolon after the preceding display)

\LL1173-1174 grammar
        `one of its quadratic twist.'
        \>
        `one of its quadratic twists.'

\L1177  \textsc{kerning}
        `$Curve25519\_Fp$' (\texttt{\$Curve25519\string\_Fp\$})
        \\\>\\
        `$\mathit{Curve25519\_Fp}$'
        (\texttt{\$\string\mathit\string{Curve25519\string\_Fp\string}\$})
        \\or\\
        `$\operatorname{Curve25519\_Fp}$'
        (\texttt{\$\string\operatorname\string{Curve25519\string\_Fp\string}\$})

\L1178  \textsc{kerning}
        same but with Twist25519\_Fp

\L1209  unclear
        `We can represent [$\F_p[\sqrt2]$] as the set $\F_p\times\F_p$
         with $\delta = 2$\textbf, in other words, the polynomial with
         coefficients in $\F_p$ modulo $X^2 - 2$.'
        \\
        This sentence isn't clear to me.
        The letter $\delta$ doesn't seem to appear elsewhere in the paper.
        It seems to mean the degree of a polynomial here---that is,
         you're discussing representing $\F_{p^2}$ by the quotient
         $\F_p[X]/(X^2 - 2)$, and in turn representing that quotient
         by the set of degree-2 polynomials.
        Maybe instead: `We can represent it by pairs
         $(a, b) \in \F_p\times\F_p$ representing the coset of
         $a + b X$ in the quotient $\F_p[X]/(X^2 - 2)$.'

\LL1210-1211 grammar
        `In a similar way as for $\F_p$ we use Module in Coq.'
        \>
        `As we did for $\F_p$, we use a module in Coq.'

\L1229  style
        `Similarly as in $\F_p$'
        \>
        `As in $\F_p$'

\LL1232-1233 grammar
        `abbreviated as $a$ without confusions.'
        \>
        `abbreviated as $a$ without confusion.'

\D V.18 style
        I'd typeset this as:

        \begin{quotation}
          Define the following functions:
%
          \begin{align*}
            \phi_c&\colon M_{486662,1}(\F_p) \to M_{486662,1}(\F_{p^2}),
            & (x, y) &\mapsto \bigl((x,0), (y,0)\bigr); \\
            \phi_t&\colon M_{486662,2}(\F_p) \to M_{486662,1}(\F_{p^2}),
            & (x, y) &\mapsto \bigl((x,0), (0,y)\bigr); \\
            \psi&\colon \F_{p^2} \to \F_p,
            & (x, y) &\mapsto x.
          \end{align*}
        \end{quotation}
%
        or
%
        \begin{quotation}
          Define the functions
           $\phi_c\colon M_{486662,1}(\F_p) \to M_{486662,1}(\F_{p^2})$,
           $\phi_t\colon M_{486662,2}(\F_p) \to M_{486662,1}(\F_{p^2})$,
           and
           $\psi\colon \F_{p^2} \to \F_p$
           by
%
          \begin{align*}
            \phi_c\bigl((x, y)\bigr) &= \bigl((x,0), (y,0)\bigr) \\
            \phi_t\bigl((x, y)\bigr) &= \bigl((x,0), (0,y)\bigr) \\
            \psi\bigl((x, y)\bigr) &= x.
          \end{align*}
        \end{quotation}

        That said, the use of tuples here $(x, y)$ to sometimes mean
         a point on the curve and sometimes mean an element of the
         quadratic field extension is a little confusing.
        Maybe you could use a tag like $(x, y)_{p^2}$ \vs $(x, y)_M$,
         or a different delimiter, or different letters?

\end{description}

\end{document}
