\section{Proving equivalence of X25519 in C and Coq}
\label{sec:C-Coq}

In this section we prove the following theorem:
% In this section we outline the structure of our proofs of the following theorem:

\begin{informaltheorem}
The implementation of X25519 in TweetNaCl (\TNaCle{crypto_scalarmult}) matches
the specifications of RFC~7748~\cite{rfc7748} (\Coqe{RFC}).
\end{informaltheorem}

More formally:
\begin{lstlisting}[language=Coq]
Theorem body_crypto_scalarmult:
  (* VST boiler plate. *)
  semax_body
    (* Clight translation of TweetNaCl. *)
    Vprog
    (* Hoare triples for fct calls. *)
    Gprog
    (* fct we verify. *)
    f_crypto_scalarmult_curve25519_tweet
    (* Our Hoare triple, see below. *)
    crypto_scalarmult_spec.
\end{lstlisting}

% We first describe the global structure of our proof (\ref{subsec:proof-structure}).
Using our formalization of RFC~7748 (\sref{sec:Coq-RFC}) we specify the Hoare
triple before proving its correctness with the VST (\ref{subsec:with-VST}).
We provide an example of equivalence of operations over different number
representations (\ref{subsec:num-repr-rfc}).
% Then, we describe efficient techniques used in some of our more complex proofs (\ref{subsec:inversions-reflections}).
