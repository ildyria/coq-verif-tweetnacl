\subsection{TweetNaCl specifics}
\label{subsec:Number-TweetNaCl}

As its name suggests, TweetNaCl aims for code compactness (\emph{``a crypto library in 100 tweets''}).
As a result it uses a few defines and typedefs to gain precious bytes while
still remaining human-readable.
\begin{lstlisting}[language=Ctweetnacl,stepnumber=0]
#define FOR(i,n) for (i = 0;i < n;++i)
#define sv static void
typedef unsigned char u8;
typedef long long i64;
\end{lstlisting}

TweetNaCl functions take pointers as arguments. By convention the first one
points to the output array. It is then followed by the input arguments.

Due to some limitations of the VST, indexes used in \TNaCle{for} loops have to be
of type \TNaCle{int} instead of \TNaCle{i64}. We change the code to allow our
proofs to carry through. We believe this does not affect the correctness of the
original code. A complete diff of our modifications to TweetNaCl can be found in
Appendix~\ref{verified-C-and-diff}.


\subsection{X25519 in TweetNaCl}
\label{subsec:X25519-TweetNaCl}

We now describe the implementation of X25519 in TweetNaCl.

\subheading{Arithmetic in \Ffield.}
In X25519, all computations are performed in $\F{p}$.
Throughout the computation, elements of that field
are represented in radix $2^{16}$,
i.e., an element $A$ is represented as $(a_0,\dots,a_{15})$,
with $A = \sum_{i=0}^{15}a_i2^{16i}$.
The individual ``limbs'' $a_i$ are represented as
64-bit \TNaCle{long long} variables:
\begin{lstlisting}[language=Ctweetnacl,stepnumber=0]
typedef i64 gf[16];
\end{lstlisting}

The conversion from the input byte array to this representation in radix
$2^{16}$ is done with the \TNaCle{unpack25519} function.

The radix-$2^{16}$ representation in limbs of $64$ bits is
highly redundant; for any element $A \in \Ffield$ there are
multiple ways to represent $A$ as $(a_0,\dots,a_{15})$.
This is used to avoid or delay carry handling in basic operations such as
Addition (\TNaCle{A}), subtraction (\TNaCle{Z}), multiplication (\TNaCle{M})
and squaring (\TNaCle{S}). After a multiplication, the limbs of the result
\texttt{o} are too large to be used again as input. The two calls to
\TNaCle{car25519} at the end of \TNaCle{M} propagate the carries through
the limbs.

Inverses in $\Ffield$ are computed with \TNaCle{inv25519}.
This function uses exponentiation by $p - 2 = 2^{255}-21$,
computed with the square-and-multiply algorithm.

\TNaCle{sel25519} implements a constant-time conditional swap (\texttt{CSWAP}) by
applying a mask between two fields elements.

Finally, we need the \TNaCle{pack25519} function,
which converts from the internal redundant radix-$2^{16}$
representation to a unique byte array representing an
integer in $\{0,\dots,p-1\}$ in little-endian format.

This function is considerably more complex as it needs to convert
to a \emph{unique} representation, i.e., also fully reduce modulo
$p$ and remove the redundancy of the radix-$2^{16}$ representation.

The C definition of those functions are available in
Appendix \ref{verified-C-and-diff}.

\subheading{The Montgomery ladder.}
With these low-level arithmetic and helper functions defined,
we can now turn our attention to the core of the X25519 computation:
the \TNaCle{crypto_scalarmult} API function of TweetNaCl,
which is implemented through the Montgomery ladder.

\begin{lstlisting}[language=Ctweetnacl]
int crypto_scalarmult(u8 *q,
                      const u8 *n,
                      const u8 *p)
{
  u8 z[32];
  i64 r;
  int i;
  gf x,a,b,c,d,e,f;
  FOR(i,31) z[i]=n[i];
  z[31]=(n[31]&127)|64;
  z[0]&=248;
  unpack25519(x,p);
  FOR(i,16) {
    b[i]=x[i];
    d[i]=a[i]=c[i]=0;
  }
  a[0]=d[0]=1;
  for(i=254;i>=0;--i) {
    r=(z[i>>3]>>(i&7))&1;
    sel25519(a,b,r);
    sel25519(c,d,r);
    A(e,a,c);
    Z(a,a,c);
    A(c,b,d);
    Z(b,b,d);
    S(d,e);
    S(f,a);
    M(a,c,a);
    M(c,b,e);
    A(e,a,c);
    Z(a,a,c);
    S(b,a);
    Z(c,d,f);
    M(a,c,_121665);
    A(a,a,d);
    M(c,c,a);
    M(a,d,f);
    M(d,b,x);
    S(b,e);
    sel25519(a,b,r);
    sel25519(c,d,r);
  }
  inv25519(c,c);
  M(a,a,c);
  pack25519(q,a);
  return 0;
}
\end{lstlisting}

Also note that lines 10 \& 11 represent the ``clamping'' operation.
