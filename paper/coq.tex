~\\
~\\
~\\
\subsection{Coq definitions}
\label{appendix:coq}

\subsubsection{Montgomery Ladder}
\label{subsubsec:coq-ladder}
~
Generic definition of the ladder:

\begin{lstlisting}[language=Coq]
(* Typeclass to encapsulate the operations *)
Class Ops (T T': Type) (Mod: T -> T):=
{
  +   : T -> T -> T;           (* Add           *)
  *   : T -> T -> T;           (* Mult          *)
  -   : T -> T -> T;           (* Sub           *)
  x^2  : T -> T;                (* Square        *)
  C_0 : T;                     (* Constant 0    *)
  C_1 : T;                     (* Constant 1    *)
  C_121665: T;                 (* const (a-2)/4 *)
  Sel25519: Z -> T -> T -> T; (* CSWAP         *)
  Getbit: Z -> T' -> Z;       (* ith bit       *)
}.

Fixpoint montgomery_rec (m : nat) (z : T')
(a: T) (b: T) (c: T) (d: T) (e: T) (f: T) (x: T) :
(* a: x2              *)
(* b: x3              *)
(* c: z2              *)
(* d: z3              *)
(* e: temporary  var  *)
(* f: temporary  var  *)
(* x: x1              *)
(T * T * T * T * T * T) :=
match m with
| 0%nat => (a,b,c,d,e,f)
| S n =>
  let r := Getbit (Z.of_nat n) z in
    (* k_t = (k >> t) & 1                       *)
    (* swap <- k_t                              *)
  let (a, b) := (Sel25519 r a b, Sel25519 r b a) in
    (* (x_2, x_3) = cswap(swap, x_2, x_3)            *)
  let (c, d) := (Sel25519 r c d, Sel25519 r d c) in
    (* (z_2, z_3) = cswap(swap, z_2, z_3)            *)
  let e := a + c in  (* A = x_2 + z_2              *)
  let a := a - c in  (* B = x_2 - z_2              *)
  let c := b + d in  (* C = x_3 + z_3              *)
  let b := b - d in  (* D = x_3 - z_3              *)
  let d := e ^2 in    (* AA = A^2                  *)
  let f := a ^2 in    (* BB = B^2                  *)
  let a := c * a in  (* CB = C * B               *)
  let c := b * e in  (* DA = D * A               *)
  let e := a + c in  (* x_3 = (DA + CB)^2          *)
  let a := a - c in  (* z_3 = x_1 * (DA - CB)^2     *)
  let b := a ^2 in    (* z_3 = x_1 * (DA - CB)^2     *)
  let c := d - f in  (* E = AA - BB             *)
  let a := c * C_121665 in
                     (* z_2 = E * (AA + a24 * E) *)
  let a := a + d in  (* z_2 = E * (AA + a24 * E) *)
  let c := c * a in  (* z_2 = E * (AA + a24 * E) *)
  let a := d * f in  (* x_2 = AA * BB            *)
  let d := b * x in  (* z_3 = x_1 * (DA - CB)^2    *)
  let b := e ^2 in    (* x_3 = (DA + CB)^2         *)
  let (a, b) := (Sel25519 r a b, Sel25519 r b a) in
    (* (x_2, x_3) = cswap(swap, x_2, x_3)           *)
  let (c, d) := (Sel25519 r c d, Sel25519 r d c) in
    (* (z_2, z_3) = cswap(swap, z_2, z_3)           *)
  montgomery_rec n z a b c d e f x
end.

Definition get_a (t:(T * T * T * T * T * T)) : T :=
match t with
  (a,b,c,d,e,f) => a
end.
Definition get_c (t:(T * T * T * T * T * T)) : T :=
match t with
  (a,b,c,d,e,f) => c
end.
\end{lstlisting}

\subsubsection{ZCrypto\_Scalarmult}
\label{subsubsec:ZCryptoScalarmult}
~
Instanciation of the Class \Coqe{Ops} with operations over \Z and modulo \p.
\begin{lstlisting}[language=Coq]
Definition modP (x:Z) : Z :=
  Z.modulo x (Z.pow 2 255 - 19).

(* Encapsulate in a module. *)
Module Mid.
  (* shift to the right by n bits *)
  Definition getCarry (n:Z) (m:Z) : Z :=
    Z.shiftr m n.

  (* logical and with n ones *)
  Definition getResidue (n:Z)  (m:Z) : Z :=
    Z.land n (Z.ones n).

  Definition car25519 (n:Z) : Z  :=
    38 * getCarry 256 n +  getResidue 256 n.
  (* The carry operation is invariant under modulo *)
  Lemma Zcar25519_correct:
    forall (n:Z), n:GF = (Mid.car25519 n) :GF.

  (* Define Mid.A, Mid.M ... *)
  Definition A a b := Z.add a b.
  Definition M a b :=
    car25519 (car25519 (Z.mul a b)).
  Definition Zub a b := Z.sub a b.
  Definition Sq a := M a a.
  Definition C_0 := 0.
  Definition C_1 := 1.
  Definition C_121665 := 121665.
  Definition Sel25519 (b p q:Z) :=
    if (Z.eqb b 0) then p else q.

  Definition getbit (i:Z) (a: Z) :=
    if (Z.ltb a 0) then
      0
    else if (Z.ltb i 0) then
        Z.land a 1
    else
        Z.land (Z.shiftr a i) 1.
End Mid.

(* Packing is applying a modulo p *)
Definition ZPack25519 n :=
  Z.modulo n (Z.pow 2 255 - 19).

(* And with 255 ones *)
(* unset last 3 bits *)
(* set bit 254       *)
Definition Zclamp (n : Z) : Z :=
  (Z.lor
    (Z.land n (Z.land (Z.ones 255) (-8)))
    (Z.shiftl 64 (31 * 8))).

(* x^{p - 2} *)
Definition ZInv25519 (x:Z) : Z :=
  Z.pow x (Z.pow 2 255 - 21).

(* instanciate over Z *)
Instance Z_Ops : (Ops Z Z modP) := {}.
Proof.
  apply Mid.A.        (* instanciate +       *)
  apply Mid.M.        (* instanciate *       *)
  apply Mid.Zub.      (* instanciate -       *)
  apply Mid.Sq.       (* instanciate x^2      *)
  apply Mid.C_0.      (* instanciate Const 0 *)
  apply Mid.C_1.      (* instanciate Const 1 *)
  apply Mid.C_121665. (* instanciate (a-2)/4 *)
  apply Mid.Sel25519. (* instanciate CSWAP   *)
  apply Mid.getbit.   (* instanciate ith bit *)
Defined.

(* instanciate montgomery_rec with Z_Ops *)
Definition ZCrypto_Scalarmult n p :=
  let t := montgomery_rec
    255               (* iterate 255 times *)
    (Zclamp n)        (* clamped n         *)
    1                 (* x_2                *)
    (ZUnpack25519 p)  (* x_3                *)
    0                 (* z_2                *)
    1                 (* z_3                *)
    0                 (* dummy             *)
    0                 (* dummy             *)
    (ZUnpack25519 p)  (* x_1                *) in
  let a := get_a t in
  let c := get_c t in
  ZPack25519 (Z.mul a (ZInv25519 c)).
\end{lstlisting}

\subsubsection{CSM}
\label{subsubsec:CryptoScalarmult}
~
\begin{lstlisting}[language=Coq]
Definition Crypto_Scalarmult n p :=
let t := montgomery_rec
  255               (* iterate 255 times *)
  (clamp n)         (* clamped n         *)
  Low.C_1           (* x_2                *)
  (Unpack25519 p)   (* x_3                *)
  Low.C_0           (* z_2                *)
  Low.C_1           (* z_3                *)
  Low.C_0           (* dummy             *)
  Low.C_0           (* dummy             *)
  (Unpack25519 p)   (* x_1                *) in
let a := get_a t in
let c := get_c t in
Pack25519 (Low.M a (Inv25519 c)).

Definition CSM := Crypto_Scalarmult.
\end{lstlisting}

\subsubsection{Equivalence between For Loops}
\label{subsubsec:for}
~
\begin{lstlisting}[language=Coq]
Variable T: Type.
Variable g: nat -> T -> T.

Fixpoint rec_fn (n:nat) (s:T) :=
  match n with
  | 0 => s
  | S n => rec_fn n (g n s)
  end.

Fixpoint rec_fn_rev_acc (n:nat) (m:nat) (s:T) :=
  match n with
  | 0 => s
  | S n => g (m - n - 1) (rec_fn_rev_acc n m s)
  end.

Definition rec_fn_rev (n:nat) (s:T) :=
  rec_fn_rev_acc n n s.

Lemma Tail_Head_equiv :
  forall (n:nat) (s:T),
  rec_fn n s = rec_fn_rev n s.
\end{lstlisting}
