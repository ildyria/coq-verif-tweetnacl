\section{Conclusion}
\label{sec:Conclusion}

Any formal system relies on a trusted base. In this section we describe our
chain of trust.

\subheading{Trusted Code Base of the proof.}
Our proof relies on a trusted base, i.e. a foundation of definitions that must be
correct. One should not be able to prove a false statement in that system, \eg by
proving an inconsistency.

In our case we rely on:
\begin{itemize}
      \item \textbf{Calculus of Inductive Constructions}. The intuitionistic logic
            used by Coq must be consistent in order to trust the proofs. As an axiom,
            we assume that the functional extensionality is also consistent with that logic.
            $$\forall x, f(x) = g(x) \implies f = g$$
            \begin{lstlisting}[language=Coq,belowskip=-0.25 \baselineskip]
Lemma f_ext: forall (A B:Type),
  forall (f g: A -> B),
  (forall x, f(x) = g(x)) -> f = g.
\end{lstlisting}

      \item \textbf{Verifiable Software Toolchain}. This framework developed at
            Princeton allows a user to prove that a Clight code matches pure Coq
            specification.

      \item \textbf{CompCert}. When compiling with CompCert we only need to trust
            CompCert's {assembly} semantics, as the compilation chain has been formally proven correct.
            However, when compiling with other C compilers like Clang or GCC, the
            whole code base of these compilers becomes part of the TCB.

      \item \textbf{\texttt{clightgen}}. The tool translating from {C} to
                  {Clight}, the first step of the CompCert compilation.
            This compilation step is not covered by the proofs of CompCert
            and VST requires Clight input. For example, VST does not support the direct verification of
            \texttt{o[i] = a[i] + b[i]}, which \texttt{clightgen} translates to
            \begin{lstlisting}[language=Ctweetnacl,stepnumber=0,belowskip=-0.5 \baselineskip]
aux1 = a[i]; aux2 = b[i];
o[i] = aux1 + aux2;
\end{lstlisting}
            The \texttt{-normalize} flag is taking care of this
            rewriting and factors out assignments from inside subexpressions.
            % The trust of the proof relies on a correct translation from the
            % initial version of \emph{TweetNaCl} to \emph{TweetNaClVerifiableC}.
            % The changes required for C code to make it verifiable are now minimal.

      \item Finally, we must trust the \textbf{Coq kernel} and its
            associated libraries; the \textbf{Ocaml compiler} on which we compiled Coq;
            the \textbf{Ocaml Runtime} and the \textbf{CPU}. Those are common to all proofs
            done with this architecture \cite{2015-Appel,coq-faq}.
\end{itemize}

\subheading{Corrections in TweetNaCl.}
As a result of this verification, we removed superfluous code.
Indeed indexes 17 to 79 of the \TNaCle{i64 x[80]} intermediate variable of
\TNaCle{crypto_scalarmult} were adding unnecessary complexity to the code,
we removed them.

Wu and Donenfeld brought to our attention that the original
\TNaCle{car25519} function carried a risk of undefined behavior if \texttt{c}
is a negative number.
\begin{lstlisting}[language=Ctweetnacl,stepnumber=0]
c=o[i]>>16;
o[i]-=c<<16; // c < 0 = UB !
\end{lstlisting}
We replaced this statement with a logical \texttt{and}, proved correctness,
and thus solved this problem.
\begin{lstlisting}[language=Ctweetnacl,stepnumber=0]
o[i]&=0xffff;
\end{lstlisting}

Aside from this modifications, all subsequent alterations to the TweetNaCl code%
---such as the type change of loop indexes (\TNaCle{int} instead of \TNaCle{i64})---%
were required for VST to step through the code properly. We believe that those
adjustments do not impact the trust of our proof.

We contacted the authors of TweetNaCl and expect that the changes described
above will soon be integrated in a new version of the library.


% Do we want to say that ?

% \subheading{Verification Effort.}
% In addition to the time required to get familiar with
% research software, we faced a few bugs which we reported
% to the developers of VST to get them fixed.
% It is very hard to work with a tool without being involved
% in the development loop. Additionally newer versions often
% broke some of our proofs and it was often needed to adapt
% to the changes.
% As a result we do not believe the metric person-month to be
% a good representation of the verification effort.

\subheading{Lessons learned.}
Most efforts in the area of high-assurance crypto are carried out
by teams who at the same time work on tools and proofs and often
even co-develop the implementations with the proofs.
In this effort we set out to verify pre-existing software,
written in a not particularly verification-friendly language
using a set of tools (VST and Coq) whose development we are not
actively involved in.

TweeNaCl comes with a claim of verifiability, but it became clear
rather quickly that this claim is only based on the overall simplicity
of the library and not supported by code written carefully such that it can
efficiently be verified with existing tools. The difference between
our verified version of TweetNaCl and the original TweetNaCl in
Appendix~\ref{verified-C-and-diff} gives an idea of some minimal
changes we had to apply to work with VST; many more small changes
would have made the proof much easier and more elegant. As one
example, in \TNaCle{pack25519} the substraction of $p$ and the carry
propagation are done in a single \TNaCle{for} loop;
had they been splitted into two loops, the final result would have been the same with a
verification effort significantly lessen.

There were many positive lessons to be learned from this verification effort;
most importantly that it is possible to prove ``legacy'' cryptographic
software written in C correct without having to co-develop proofs
and tools. However, we also learned that it is still necessary to
understand to some extent how these tools (in particular VST)
work under the hood.
VST is a collection of lemmas and proof tactics; the idea is
to expose the user only to the tactics and hide the details of
the underlying lemmas.
At least in the VST versions we worked with,
this approach did not quite work and at various stages in
the proofs we had to look into the underlying lemmas.
This was due to the provided tactics not terminating,
for example in the last few steps \coqe{pack25519}'s VST proof.
Some struggle with VST also taught us another very pleasant lesson,
namely that the VST development team is very responsive and helpful.
Various of our issues were sorted out with their help and we hope
that some of the experience we brought in also helped improve VST.

\subheading{Extending our work.}
The high-level definition (\sref{sec:maths}) can easily be ported to any
other Montgomery curves and with it the proof of the ladder's correctness
assuming the same formulas are used.
In addition to the curve equation, the field \F{p} would need to be redefined
as $p=2^{255}-19$ is hard-coded in order to speed up some proofs.

With respect to the C code verification (\sref{sec:C-Coq}), the extension of
the verification effort to Ed25519 would make directly use of the low-level
arithmetic. The ladder steps formula being different this would require a high
level verification similar to \tref{thm:montgomery-ladder-correct};
also, a full correctness verification of Ed25519 signatures would require
verifying correctness of SHA-512.

The verification of \eg X448~\cite{cryptoeprint:2015:625,rfc7748} in C would
require the adaptation of most of the low-level arithmetic (mainly the
multiplication, carry propagation and reduction).
Once the correctness and bounds of the basic operations are established,
reproving the full ladder would make use of our generic definition.

\subheading{A complete proof.}
We provide a mechanized formal proof of the correctness of the X25519
implementation in TweetNaCl from C up the mathematical definitions with a single tool.
We first formalized X25519 from RFC~7748~\cite{rfc7748} in Coq.
We then proved that TweetNaCl's implementation of X25519 matches our formalization.
In a second step we extended the Coq library for elliptic curves \cite{BartziaS14}
by Bartzia and Strub to support Montgomery curves.
Using this extension we proved that the X25519 from the RFC matches the
mathematical definitions as given in~\cite[Sec.~2]{Ber06}.
Therefore in addition to proving the mathematical correctness of TweetNaCl,
we also increases the trust of other works such as
\cite{zinzindohoue2017hacl,Erbsen2016SystematicSO} which rely on RFC~7748.
