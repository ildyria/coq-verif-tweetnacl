\subsection{Coq, separation logic, and VST}
\label{subsec:Coq-VST}

Coq~\cite{coq-faq} is an interactive theorem prover based on type theory. It
provides an expressive formal language to write mathematical definitions,
algorithms, and theorems together with their proofs. It has been used in the proof
of the four-color theorem~\cite{gonthier2008formal} and it is also the system
underlying the CompCert formally verified C compiler~\cite{Leroy-backend}.
Unlike systems like F*~\cite{DBLP:journals/corr/BhargavanDFHPRR17},
Coq does not rely on an SMT solver in its trusted code base.
It uses its type system to verify the applications of hypotheses,
lemmas, and theorems~\cite{Howard1995-HOWTFN}.

Hoare logic is a formal system which allows reasoning about programs.
It uses triples such as
$$\{{\color{doc@lstnumbers}\textbf{Pre}}\}\texttt{~Prog~}\{{\color{doc@lstdirective}\textbf{Post}}\}$$
where ${\color{doc@lstnumbers}\textbf{Pre}}$ and ${\color{doc@lstdirective}\textbf{Post}}$
are assertions and \texttt{Prog} is a fragment of code.
It is read as
``when the precondition  ${\color{doc@lstnumbers}\textbf{Pre}}$ is met,
executing \texttt{Prog} will yield postcondition ${\color{doc@lstdirective}\textbf{Post}}$''.
We use compositional rules to prove the truth value of a Hoare triple.
For example, here is the rule for sequential composition:
\begin{prooftree}
  \AxiomC{$\{P\}C_1\{Q\}$}
  \AxiomC{$\{Q\}C_2\{R\}$}
  \LeftLabel{Hoare-Seq}
  \BinaryInfC{$\{P\}C_1;C_2\{R\}$}
\end{prooftree}
Separation logic is an extension of Hoare logic which allows reasoning about
pointers and memory manipulation. This logic enforces strict conditions on the
memory shared such as being disjoint.
We discuss this limitation further in \sref{subsec:with-VST}.

The Verified Software Toolchain (VST)~\cite{cao2018vst-floyd} is a framework
which uses separation logic (proven correct with respect to CompCert semantics)
to prove the functional correctness of C programs.
The first step consists of translating the source code into Clight,
an intermediate representation used by CompCert.
For such purpose one uses the parser of CompCert called \texttt{clightgen}.
In a second step one defines the Hoare triple representing the specification of
the piece of software one wants to prove. Then using VST, one uses a strongest
postcondition approach to prove the correctness of the triple.
This can be seen as a forward symbolic execution of the program.
