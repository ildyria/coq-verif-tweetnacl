Then using ``Reflections'' techniques (chapter 15 in \cite{CpdtJFR}), we prove
the functional correctness of the inversion over \Zfield.
\begin{lemma}
\label{cor:inv_comput_field}
\Coqe{Inv25519} computes an inverse in \Zfield.
\end{lemma}
This statement is formalized as
\begin{lstlisting}[language=Coq]
Corollary Inv25519_Zpow_GF : forall (g:list Z),
  length g = 16 ->
  Z16.lst (Inv25519 g) :GF  =
    (pow (Z16.lst g) (2^255-21)) :GF.
\end{lstlisting}

\begin{sloppypar}
By using each function \coqe{Low.M}; \coqe{Low.A}; \coqe{Low.Sq}; \coqe{Low.Zub};
\coqe{Unpack25519}; \coqe{clamp}; \coqe{Pack25519}; \coqe{Inv25519}; \coqe{car25519};
\coqe{montgomery_rec}, we defined in Coq \coqe{Crypto_Scalarmult} and with VST
proved it mimicks the exact behavior of X25519 in TweetNaCl.
\end{sloppypar}

\begin{sloppypar}
By proving that each function \coqe{Low.M}; \coqe{Low.A}; \coqe{Low.Sq};
\coqe{Low.Zub}; \coqe{Unpack25519}; \coqe{clamp}; \coqe{Pack25519};
\coqe{Inv25519}; \coqe{car25519} behave over \coqe{list Z} as their equivalent
over \coqe{Z} with \coqe{:GF} (in \Zfield), we prove that given the same inputs
\coqe{Crypto_Scalarmult} performs the same computation as \coqe{RFC}.
\end{sloppypar}

% This is formalized as follows in Coq:
\begin{lstlisting}[language=Coq]
Lemma Crypto_Scalarmult_RFC_eq :
  forall (n: list Z) (p: list Z),
  Zlength n = 32 ->
  Zlength p = 32 ->
  Forall (fun x => 0 <= x /\ x < 2 ^ 8) n ->
  Forall (fun x => 0 <= x /\ x < 2 ^ 8) p ->
  Crypto_Scalarmult n p = RFC n p.
\end{lstlisting}

This proves that TweetNaCl's X25519 implementation respect RFC~7748.

\todo{add what can be reused here ?}





%% PREVIOUS TEXT BELOW.



% \subsection{Reflections, inversions and packing}
% \label{subsec:inversions-reflections}
%
% We now turn our attention to the multiplicative inverse in $\Zfield$ and techniques
% to improve the verification speed of complex formulas.
%
% \subheading{Inversion in \Zfield.}
% We define a Coq version of the inversion mimicking
% the behavior of \TNaCle{inv25519} (see below) over \Coqe{list Z}.
% \begin{lstlisting}[language=Ctweetnacl]
% sv inv25519(gf o,const gf i)
% {
%   gf c;
%   int a;
%   set25519(c,i);
%   for(a=253;a>=0;a--) {
%     S(c,c);
%     if(a!=2&&a!=4) M(c,c,i);
%   }
%   FOR(a,16) o[a]=c[a];
% }
% \end{lstlisting}
% We specify this with 2 functions: a recursive \Coqe{pow_fn_rev}
% to simulate the \texttt{for} loop and a simple \Coqe{step_pow} containing the body.
% \begin{lstlisting}[language=Coq]
% Definition step_pow (a:Z)
%   (c:list Z) (g:list Z) : list Z :=
%   let c := Sq c in
%     if a <>? 2 && a <>? 4
%     then M c g
%     else c.
%
% Function pow_fn_rev (a:Z) (b:Z)
%   (c: list Z) (g: list Z)
%   {measure Z.to_nat a} : (list Z) :=
%   if a <=? 0
%     then c
%     else
%       let prev := pow_fn_rev (a - 1) b c g in
%         step_pow (b - a) prev g.
% \end{lstlisting}
% This \Coqe{Function} requires a proof of termination. It is done by proving the
% well-foundedness of the decreasing argument: \Coqe{measure Z.to_nat a}. Calling
% \Coqe{pow_fn_rev} 254 times allows us to reproduce the same behavior as the Clight definition.
% \begin{lstlisting}[language=Coq]
% Definition Inv25519 (x:list Z) : list Z :=
%   pow_fn_rev 254 254 x x.
% \end{lstlisting}
% Similarly we define the same function over $\Z$.
% \begin{lstlisting}[language=Coq]
% Definition step_pow_Z (a:Z) (c:Z) (g:Z) : Z :=
%   let c := c * c in
%   if a <>? 2 && a <>? 4
%     then c * g
%     else c.
%
% Function pow_fn_rev_Z (a:Z) (b:Z) (c:Z) (g: Z)
%   {measure Z.to_nat a} : Z :=
%   if (a <=? 0)
%     then c
%     else
%       let prev := pow_fn_rev_Z (a - 1) b c g in
%         step_pow_Z (b - a) prev g.
%
% Definition Inv25519_Z (x:Z) : Z :=
%   pow_fn_rev_Z 254 254 x x.
% \end{lstlisting}
% By using \lref{lemma:mult_correct}, we prove their equivalence in $\Zfield$.
% \begin{lemma}
% \label{lemma:Inv_equivalence}
% The function \coqe{Inv25519} over list of integers computes the same
% result at \coqe{Inv25519_Z} over integers in \Zfield.
% \end{lemma}
% This is formalized in Coq as follows:
% \begin{lstlisting}[language=Coq]
% Lemma Inv25519_Z_GF : forall (g:list Z),
%   length g = 16 ->
%   (Z16.lst (Inv25519 g)) :GF =
%   (Inv25519_Z (Z16.lst g)) :GF.
% \end{lstlisting}
%
% In TweetNaCl, \TNaCle{inv25519} computes an inverse in $\Zfield$.
% It uses Fermat's little theorem by raising to the power of $2^{255}-21$ with a
% square-and-multiply algorithm. The binary representation
% of $p-2$ implies that every step does a multiplications except for bits 2 and 4.
% To prove the correctness of the result we could use multiple strategies such as:
% \begin{itemize}
%   \item Proving it is a special case of square-and-multiply algorithm applied to $2^{255}-21$.
%   \item Unrolling the for loop step-by-step and applying the equalities
%   $x^a \times x^b = x^{(a+b)}$ and $(x^a)^2 = x^{(2 \times a)}$. We prove that
%   the resulting exponent is $2^{255}-21$.
% \end{itemize}
% We use the second method because it is simpler. However, it requires us to
% apply the unrolling and exponentiation formulas 255 times. This could be automated
% in Coq with tacticals such as \Coqe{repeat}, but it generates a proof object which
% will take a long time to verify.
%
% \subheading{Reflections.}
% In order to speed up the verification we use a
% technique called ``Reflection''. It provides us with flexibility, \eg we don't
% need to know the number of times nor the order in which the lemmas needs to be
% applied (chapter 15 in \cite{CpdtJFR}).
%
% The idea is to \emph{reflect} the goal into a decidable environment.
% We show that for a property $P$, we can define a decidable Boolean property
% $P_{bool}$ such that if $P_{bool}$ is \Coqe{true} then $P$ holds.
% $$\text{\textit{reify\_P}} : P_{bool} = \text{\textit{true}} \implies P$$
% By applying \textit{reify\_P} on $P$ our goal becomes $P_{bool} = true$.
% We then compute the result of $P_{bool}$. If the decision goes well we are
% left with the tautology $\text{\textit{true}} = \text{\textit{true}}$.
%
% With this technique we prove the functional correctness of the inversion over \Zfield.
% \begin{lemma}
% \label{cor:inv_comput_field}
% \Coqe{Inv25519} computes an inverse in \Zfield.
% \end{lemma}
% This statement is formalized as
% \begin{lstlisting}[language=Coq]
% Corollary Inv25519_Zpow_GF : forall (g:list Z),
%   length g = 16 ->
%   Z16.lst (Inv25519 g) :GF  =
%     (pow (Z16.lst g) (2^255-21)) :GF.
% \end{lstlisting}
%
% This reflection technique is also used where proofs requires some computing
% over a small and finite domain of variables to test e.g. for all $i$ such that
% $0 \le i < 16$.
% Using reflection we prove that we can split the for loop in \TNaCle{pack25519}
% into two parts.
% \begin{lstlisting}[language=Ctweetnacl]
% for(i=1;i<15;i++) {
%   m[i]=t[i]-0xffff-((m[i-1]>>16)&1);
%   m[i-1]&=0xffff;
% }
% \end{lstlisting}
% The first loop computes the subtraction, and the second applies the carries.
% \begin{lstlisting}[language=Ctweetnacl]
% for(i=1;i<15;i++) {
%   m[i]=t[i]-0xffff
% }
%
% for(i=1;i<15;i++) {
%   m[i]=m[i]-((m[i-1]>>16)&1);
%   m[i-1]&=0xffff;
% }
% \end{lstlisting}
%
% This loop separation allows simpler proofs. The first loop is seen as the
% subtraction of \p. The resulting number represented in $\Zfield$ is invariant with
% the iteration of the second loop. This result in the proof that \TNaCle{pack25519}
% reduces modulo $\p$ and returns a number in base $2^8$.
% \begin{lstlisting}[language=Coq]
% Lemma Pack25519_mod_25519 :
%   forall (l:list Z),
%   Zlength l = 16 ->
%   Forall (fun x => -2^62 < x < 2^62) l ->
%   ZofList 8 (Pack25519 l) =
%   (Z16.lst l) mod (2^255-19).
% \end{lstlisting}
%
% By using each function \coqe{Low.M}; \coqe{Low.A}; \coqe{Low.Sq}; \coqe{Low.Zub};
% \coqe{Unpack25519}; \coqe{clamp}; \coqe{Pack25519}; \coqe{Inv25519}; \coqe{car25519}; \coqe{montgomery_rec},
% we defined a Coq definition \coqe{Crypto_Scalarmult} mimicking the exact behavior of X25519 in TweetNaCl.
%
% By proving that each function \coqe{Low.M}; \coqe{Low.A}; \coqe{Low.Sq}; \coqe{Low.Zub};
% \coqe{Unpack25519}; \coqe{clamp}; \coqe{Pack25519}; \coqe{Inv25519}; \coqe{car25519} behave over \coqe{list Z}
% as their equivalent over \coqe{Z} with \coqe{:GF} (in \Zfield), we prove that given the same inputs \coqe{Crypto_Scalarmult} performs the same computation as \coqe{RFC}.
% % This is formalized as follows in Coq:
% \begin{lstlisting}[language=Coq]
% Lemma Crypto_Scalarmult_RFC_eq :
%   forall (n: list Z) (p: list Z),
%   Zlength n = 32 ->
%   Zlength p = 32 ->
%   Forall (fun x => 0 <= x /\ x < 2 ^ 8) n ->
%   Forall (fun x => 0 <= x /\ x < 2 ^ 8) p ->
%   Crypto_Scalarmult n p = RFC n p.
% \end{lstlisting}
%
% This proves that TweetNaCl's X25519 implementation respect RFC~7748.
