\section{Introduction}
\label{sec:intro}

The Networking and Cryptography library (NaCl)~\cite{BLS12}
is an easy-to-use, high-security, high-speed cryptography library.
It uses specialized code for different platforms, which makes it rather complex and hard to audit.
TweetNaCl~\cite{BGJ+15} is a compact re-implementation in C
of the core functionalities of NaCl and is claimed to be
\emph{``the first cryptographic library that allows correct functionality
to be verified by auditors with reasonable effort''}~\cite{BGJ+15}.
The original paper presenting TweetNaCl describes some effort to support
this claim, for example, formal verification of memory safety, but does not actually
prove correctness of any of the primitives implemented by the library.

One core component of TweetNaCl (and NaCl) is the key-exchange protocol X25519~\cite{rfc7748}.
This protocol is being used by a wide variety of applications~\cite{things-that-use-curve25519}
such as SSH, Signal Protocol, Tor, Zcash, and TLS to establish a shared secret over
an insecure channel.
The X25519 key exchange protocol is a an $x$-coordinate only
elliptic-curve Diffie-Hellman key-exchange using the Montgomery
curve $E: y^2 = x^3 + 486662 x^2 + x$ over the field $\F{\p}$.
Note that originally, the name ``Curve25519'' referred to the key exchange protocol,
but Bernstein suggested to rename the scheme to X25519 and to use the name
X25519 for the protocol and Curve25519 for the underlying elliptic curve~\cite{Ber14}.
We make use of this updated terminology in this paper.

\subheading{Contribution of this paper}
We provide a mechanized formal proof of the correctness of the X25519
implementation in TweetNaCl.
This proof is done in two steps:
We first prove equivalence of the C implementation of X25519
to an implementation in Coq~\cite{coq-faq}.
This part of the proof uses the Verifiable Software Toolchain (VST)~\cite{2012-Appel}
to establish a link between C and Coq.
VST uses Separation logic~\cite{1969-Hoare,Reynolds02separationlogic}
to show that the semantics of the program satisfy a functional specification in Coq.
To the best of our knowledge, this is the first time that
VST is used in the formal proof of correctness of an implementation
of an asymmetric cryptographic primitive.

In a second step we prove that the Coq implementation matches
the mathematical definition of X25519 as given in~\cite[Sec.~2]{Ber06}.
We accomplish this step of the proof by extending the Coq library
for elliptic curves~\cite{BartziaS14} by Bartzia and Strub to
support Montgomery curves (and in particular Curve25519).
This extension may be of independent interest.

\subheading{Related work.}
Two methodologies exist to get formal guarantees that software meets its specification.
This can be done either by synthesizing a formal specification and then generating
machine code by refinement, or by writing a specification and verifying that the
implementation satisfies it.

% One of the earliest example of this first approach is the
% B-method~\cite{Abrial:1996:BAP:236705} in 1986.
Using the synthesis approach, Zinzindohou{\'{e}} et al. wrote a verified extensible
library of elliptic curves~\cite{Zinzindohoue2016AVE} in F*~\cite{DBLP:journals/corr/BhargavanDFHPRR17}.
This served as ground work for the cryptographic library HACL*~\cite{zinzindohoue2017hacl}
used in the NSS suite from Mozilla.

Coq not only allows verification but also synthesis~\cite{CpdtJFR}.
Erbsenm; Philipoom; Gross; and Chlipala et al. make use of it to have
correct-by-construction finite field arithmetic~\cite{Philipoom2018CorrectbyconstructionFF}.
In such way, they synthesized~\cite{Erbsen2016SystematicSO} certified elliptic curve
implementations~\cite{Erbsen2017CraftingCE}. These are now being used in
BoringSSL~\cite{fiat-crypto}.

The verification approach has been used to prove the correctness of OpenSSL
HMAC~\cite{Beringer2015VerifiedCA} and SHA-256~\cite{2015-Appel}. Compared to
that work our approach makes a complete link between the C implementation and
the formal mathematical definition of the group theory behind elliptic curves.

Synthesis approaches force the user to depend on generated code which may not
be optimal (speed, compactness...) and they require compiler tweaks in order
to achieve the desired properties. On the other hand, carefully handcrafted
optimized code will force the verifier to consider all the possible special cases
and write a low level specification of the details of the code in order to prove
the correctness of the algorithm.

It has to be noted that Curve25519 implementations have already been under scrutiny~\cite{Chen2014VerifyingCS}.
However in this paper we provide a proofs of correctness and soundness of a C program up to
the theory of elliptic curves.

\subheading{Reproducing the proofs.}
To maximize reusability of our results we placed the code of our formal proof
presented in this paper into the public domain. They are available at XXXXXXX
with instructions of how to compile and verify our proof.
A description of the content of the archive is provided in
Appendix~\ref{appendix:proof-folders}.

\subheading{Organization of this paper.}
\sref{sec:preliminaries} give the necessary background on Curve25519 and X25519
implementations and a brief explanation of how formal verification works.
\sref{sec:Coq-RFC} provides our formalization of RFC~7748~\cite{rfc7748}'s X25519.
\sref{sec:C-Coq} provides the specification of TweetNaCl's X25519 and some of the
proofs techniques used to show the correctness with respect to RFC~7748~\cite{rfc7748}.
\sref{sec:maths} describes our extension of the formal library by Bartzia
and Strub and the correctness of X25519 implementation with respect to Bernstein's
specifications~\cite{Ber14}.
And lastly in \sref{sec:Conclusion} we discuss the trusted code base of this proof.


% Five years ago:
% \url{https://www.imperialviolet.org/2014/09/11/moveprovers.html}
% \url{https://www.imperialviolet.org/2014/09/07/provers.html}

% \section{Related Works}
%
% \begin{itemize}
%   \item HACL*
%   \item Proving SHA-256 and HMAC
%   \item \url{http://www.iis.sinica.edu.tw/~bywang/papers/ccs17.pdf}
%   \item \url{http://www.iis.sinica.edu.tw/~bywang/papers/ccs14.pdf}
%   \item \url{https://cryptojedi.org/crypto/#gfverif}
%   \item \url{https://cryptojedi.org/crypto/#verify25519}
%   \item Fiat Crypto : synthesis
% \end{itemize}
%
% Add comparison with Fiat-crypto
