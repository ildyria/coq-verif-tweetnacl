\section{Introduction}

Implementing cryptographic primitives without any bugs is hard. While tests
provides a decent code coverage, they don't cover 100\% of the possible values.

We verify the implementation in C of Curve25519 in Tweetnacl\cite{BGJ+15}.




Using the \textit{clightgen} tool from Compcert\cite{Leroy-backend}, we can
generate the \textit{semantic version} (Clight\cite{Blazy-Leroy-Clight-09}) of
the C code. Using the Floyd–Hoare logic\cite{1969-Hoare} with the Verifiable
Software Toolchain (VST)\cite{2012-Appel} we can show that the semantic of the
program is equivalent to a functionnal specification in Coq.
We can then prove that this specification is represent the scalar multiplication
on Curve25519\cite{}.

\subsection{Meet-in-the-middle Approach}

In order to prove that \texttt{crypto\_scalarmult} is computing a scalar
multiplication over the x-coordinate of a point P, we need to define multiples
levels of specifications and show equivalence between them.

\begin{enumerate}
  \item Write a high level specification (over a generic field $\mathbb{F}$).
  \item Show that the high level specification is equivalent to the
  computation of a montgomery ladder.
  \item Write a low level specification (e.g. over lists of $\mathbb{Z}$).
  \item Show that the low level specification represent the operations of
  defined C code.
  \item Show that the low level specification are equivalent to simple
  operations in $\mathbb{Z}_{2^{255}-19}$ (middle level specification).
  \item Show that the middle level specification is an instance of the high
  level one.
\end{enumerate}

The equivalence between each level, garantees us the correctness of the
implementation.
