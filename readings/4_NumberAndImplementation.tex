\section{Linking C and Coq}

In this section we describe techniques used to prove the equivalence between the
Clight description of TweetNaCl and Coq functions producing similar behaviors.

\subsection{Number representation and C implementation}

As described in Section \ref{sec:impl}, numbers in \TNaCle{gf} are represented
in base $2^16$ and we can use a direct mapping to represent that array as a list
integers in Coq. However in order to show the correctness of the basic operations,
we need to convert this number as a full integer. For this reason We define
\coqDe{ZofList : Z -> list Z -> Z}:
\begin{lstlisting}[language=CoqD]
Fixpoint ZofList {n:Z} (a:list Z) : Z :=
  match a with
  | [] => 0
  | h :: q => h + 2^n * ZofList q
  end.
Notation "Z16.lst A" := (ZofList 16 A).
\end{lstlisting}
It convert any list of integer from a base $2^n$ to an single integer. In our
case we define a notation where $n$ is $16$.

The following Coq notation is defined to convert any integer to its representant
in $\mathbb{Z}_{2^{255}-19}$.
\begin{lstlisting}[language=CoqD]
Notation "A :GF" := (A mod (2^255-19)).
\end{lstlisting}

Using these two definitions, we proved lemmas such as the correctness of the
multiplication \coqDe{M}:
\begin{coqD}
Lemma mult_GF_Zlength :
  forall (a:list Z) (b:list Z),
  Zlength a = 16 ->
  Zlength b = 16 ->
   (Z16.lst (M a b)) :GF =
   (Z16.lst a * Z16.lst b) :GF.
\end{coqD}

\subsection{Inversions and Reflections}

To prove that the \coqDe{Inv25519_Z} is computing $\mathbb{Z}_{2^{255}-19}$,
In TweetNaCl, \TNaCle{inv25519} computes an inverse in $\mathbb{Z}_{2^{255}-19}$

\coqDe{Inv25519} the inverse is computed using ht

{\red TODO: SAY WHY WE WANT TO USE REFLECTION}

Reflection proofs relies on the decidability of an expression. In our case,
we show that for a property $P$, we can define a decidable
boolean property $P_{bool}$ such that $$P_{bool} = true \implies P$$

With VST we proved that \TNaCle{inv25519} is equivalent to its Coq functional
definition \coqDe{Inv25519}.

\begin{coqD}
Definition Inv25519 (x:list Z) : (list Z) :=
  pow_fn_rev 254 254 x x.
\end{coqD}

We proved that \coqDe{pow_fn_rev} over \coqDe{list Z} is equivalent to applying
\coqDe{pow_fn_rev_Z} over \coqDe{Z}

\begin{coqD}
Lemma pow_fn_rev_Z_GF :
  forall (a:Z) (b:Z) (c:list Z) (g:list Z),
  Zlength c = 16 ->
  Zlength g = 16 ->
  (Z16.lst (pow_fn_rev a b c g)) :GF =
  (pow_fn_rev_Z a b (Z16.lst c) (Z16.lst g)) :GF .
\end{coqD}

where \coqDe{pow_fn_rev_Z} is defined in two parts: the application
(body of the for loop) and the loop itself.

\begin{coqD}
Definition step_pow_Z (a:Z) (c:Z) (g:Z) :=
  let c0 := c*c in
  if a <>? 1 && a <>? 3
    then c0*g
    else c0.
\end{coqD}

\begin{coqD}
Function pow_fn_rev_Z (a:Z) (b:Z) (c:Z) (g: Z)
  {measure Z.to_nat a} : (Z) :=
  if (a <=? 0)
    then c
    else
      let prev := pow_fn_rev_Z (a - 1) b c g in
        step_pow_Z (b - 1 - a) prev g.
\end{coqD}

We define the inversion modulo $2^{255}-19$ as an instance of \coqDe{pow_fn_rev_Z} where \coqDe{a} and \coqDe{b} are 254.

\begin{coqD}
Definition Inv25519_Z (x:Z) : Z :=
  pow_fn_rev_Z 254 254 x x.
\end{coqD}


\subsubsection{A Simple Domain Specific Language}

{\red TODO: WHY $2^{255}-21$ ???}

To prove that the \coqDe{Inv25519_Z} is computing $x^{2^{255}-21}$,
we define a Domain Specific Language:
\begin{coqD}
Inductive expr_inv :=
  | R_inv : expr_inv
  | M_inv : expr_inv -> expr_inv -> expr_inv
  | S_inv : expr_inv -> expr_inv
  | P_inv : expr_inv -> positive -> expr_inv.
\end{coqD}
An expression can be either a term, a multiplication, a squaring or a power.
This is denoted as follow:
\begin{coqD}
Fixpoint e_inv_denote (m:expr_inv) : Z :=
  match m with
  | R_inv     =>
    term_denote
  | M_inv x y =>
    (e_inv_denote x) * (e_inv_denote y)
  | S_inv x =>
    (e_inv_denote x) * (e_inv_denote x)
  | P_inv x p =>
    pow (e_inv_denote x) (Z.pos p)
  end.
\end{coqD}

We then defined \coqDe{step_inv} and \coqDe{pow_inv} to mirror the behavior of
\coqDe{step_pow_Z} and respectively \coqDe{pow_fn_rev_Z} over our new language.

\begin{coqD}
Lemma step_inv_step_pow_eq :
  forall (a:Z) (c:expr_inv) (g:expr_inv),
  e_inv_denote (step_inv a c g) =
  step_pow_Z a (e_inv_denote c) (e_inv_denote g).
\end{coqD}
\begin{coqD}
Lemma pow_inv_pow_fn_rev_eq :
  forall (a:Z) (b:Z) (c:expr_inv) (g:expr_inv),
  e_inv_denote (pow_inv a b c g) =
  pow_fn_rev_Z a b (e_inv_denote c) (e_inv_denote g).
\end{coqD}

We also define what is a formula in our language: a simple equality.
\begin{coqD}
Inductive formula_inv :=
  | Eq_inv : expr_inv -> expr_inv -> formula_inv.
\end{coqD}
This is denoted as follow:
\begin{coqD}
Definition f_inv_denote (t : formula_inv) : Prop :=
  match t with
  | Eq_inv x y => e_inv_denote x = e_inv_denote y
  end.
\end{coqD}
With such Domain Specific Language we have the equivalence between:
\begin{lstlisting}[backgroundcolor=\color{white}]
f_inv_denote
 (Eq_inv (M_inv R_inv (S_inv R_inv))
         (P_inv R_inv 3))
  = (x * x^2 = x^3)
\end{lstlisting}

\subsubsection{Deciding formulas}

In order to prove formulas, we define a decidable procedure.
We first compute the power of each side of the formula and then check their
equality.
\begin{coqD}
Fixpoint pow_expr_inv (t : expr_inv) : Z :=
  match t with
  | R_inv   => 1
  | M_inv x y =>
    (pow_expr_inv x) + (pow_expr_inv y)
  | S_inv x =>
    2 * (pow_expr_inv x)
  | P_inv x p =>
    (Z.pos p) * (pow_expr_inv x)
  end.

Definition decide_e_inv (l1 l2 : expr_inv) : bool :=
  (pow_expr_inv l1) ==? (pow_expr_inv l2).

Definition decide_f_inv (f : formula_inv) : bool :=
  match f with
  | Eq_inv x y => decide_e_inv x y
  end.
\end{coqD}
We proved that our procedure is correct: for all formulas, if
\coqDe{decide_f_inv} returns \coqDe{true}, then the denoted equality is correct.
\begin{coqD}
Lemma decide_formula_inv_impl :
  forall (f:formula_inv),
  decide_f_inv f = true ->
  f_inv_denote f.
\end{coqD}

By reification to our Domain Specific Language and then by applying
\texttt{decide\_formula\_inv\_impl} and computing \texttt{decide\_f\_inv},
we proved that \coqDe{Inv25519} is indeed computing an inverse in
modulo $2^{255}-19$.
\begin{coqD}
Theorem Inv25519_Z_correct :
  forall (x:Z),
  Inv25519_Z x = pow x (2^255-21).
\end{coqD}

{\red TODO: GIVE OTHER EXAMPLES WHERE REFLECTION IS APPLIED : e.g. PACK25519}
