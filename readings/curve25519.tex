% TEMPLATE for Usenix papers, specifically to meet requirements of
%  USENIX '05
% originally a template for producing IEEE-format articles using LaTeX.
%   written by Matthew Ward, CS Department, Worcester Polytechnic Institute.
% adapted by David Beazley for his excellent SWIG paper in Proceedings,
%   Tcl 96
% turned into a smartass generic template by De Clarke, with thanks to
%   both the above pioneers
% use at your own risk.  Complaints to /dev/null.
% make it two column with no page numbering, default is 10 point

% Munged by Fred Douglis <douglis@research.att.com> 10/97 to separate
% the .sty file from the LaTeX source template, so that people can
% more easily include the .sty file into an existing document.  Also
% changed to more closely follow the style guidelines as represented
% by the Word sample file. 

% Note that since 2010, USENIX does not require endnotes. If you want
% foot of page notes, don't include the endnotes package in the 
% usepackage command, below.

% This version uses the latex2e styles, not the very ancient 2.09 stuff.
\documentclass[letterpaper,twocolumn,10pt]{article}
\usepackage{usenix,epsfig}
%\usepackage{etex}
\usepackage[utf8]{inputenc}
\usepackage[T1]{fontenc}

\usepackage[pdf]{pstricks}
\usepackage{multido}

\newif\ifpublic
\publictrue

\newif\iffull
\fulltrue

\usepackage{amsfonts,amsmath,amscd,amssymb,array}

\usepackage{type1cm}
\newcommand{\lstsize}{\fontsize{8.5pt}{8.5pt}\selectfont}

\usepackage{algorithm}
\usepackage{algorithmic}

\usepackage{xcolor}
\definecolor{linkcolor}{rgb}{0.65,0,0}
\definecolor{citecolor}{rgb}{0,0.65,0}
\definecolor{urlcolor}{rgb}{0,0,0.65}
\usepackage[colorlinks=true, backref=page, linkcolor=linkcolor, urlcolor=urlcolor, citecolor=citecolor]{hyperref}
%\usepackage{breakurl}


\usepackage{ctable}
\usepackage{cite}

\usepackage{dsfont}

\usepackage{xspace}
\usepackage{subfig}
\usepackage{float}
\floatstyle{ruled}
\newfloat{Listing}{htb}{lst}
\floatstyle{plain}
\newfloat{Protocol}{htb}{proto}
\newcounter{subListing}
\newcounter{subListing@save}
\renewcommand{\thesubListing}{\alph{subListing}}

\renewcommand{\tabcolsep}{4pt}

\def\subheading#1{\medskip\noindent{\boldmath\textbf{#1}}~\ignorespaces}
\newcommand{\todo}[1]{\marginpar{\baselineskip0ex\rule{2,5cm}{0.5pt}\\[0ex]{\tiny\textsf{#1}}}}

 


% \newtheorem{theorem}{Theorem}[section]
% \newtheorem{lemma}[theorem]{Lemma}
% \newtheorem{corollary}[theorem]{Corollary}
% \newtheorem{dfn}[theorem]{Definition}
% \newtheorem{hypothesis}[theorem]{Hypothesis}

\newcommand\invisiblesection[1]{%
  \refstepcounter{section}
\sectionmark{#1}}

% \newenvironment{proof}[1][Proof]{\begin{trivlist}
% \item[\hskip \labelsep {\bfseries #1}]}{\end{trivlist}}
%   \newenvironment{definition}[1][Definition]{\begin{trivlist}
% \item[\hskip \labelsep {\bfseries #1}]}{\end{trivlist}}
%   \newenvironment{example}[1][Example]{\begin{trivlist}
% \item[\hskip \labelsep {\bfseries #1}]}{\end{trivlist}}
%   \newenvironment{remark}[1][Remark]{\begin{trivlist}
% \item[\hskip \labelsep {\bfseries #1}]}{\end{trivlist}}

% \newcommand{\qed}{\nobreak \ifvmode \relax \else
%   \ifdim\lastskip<1.5em \hskip-\lastskip
%   \hskip1.5em plus0em minus0.5em \fi \nobreak
% \vrule height0.75em width0.5em depth0.25em\fi}

\newcommand{\sref}[1]{Section~\ref{#1}}
\newcommand{\tref}[1]{Theorem~\ref{#1}}
\newcommand{\lref}[1]{Lemma~\ref{#1}}
\newcommand{\fref}[1]{Figure~\ref{#1}}
\newcommand{\aref}[1]{Algorithm~\ref{#1}}
\newcommand{\N}{\ensuremath{\mathbb{N}}\xspace}
\newcommand{\Z}{\ensuremath{\mathbb{Z}}\xspace}
\newcommand{\K}{\ensuremath{\mathbb{K}}\xspace}
\renewcommand{\L}{\ensuremath{\mathbb{L}}\xspace}
\newcommand{\corr}{\,\hat{=}\,}
\newcommand{\Oinf}{\ensuremath{\mathcal{O}}\xspace}
\newcommand{\F}[1]{\ensuremath{\mathbb{F}_{#1}}\xspace}
\newcommand{\mbf}{\ensuremath{\mathbf}}
\newcommand{\ie}{{\textit{i.e.}},\;}
\newcommand{\eg}{{\textit{e.g.}},\;}
\newcommand{\p}{\ensuremath{2^{255}-19}}
\newcommand{\Zfield}{\ensuremath{\mathbb{Z}_{\p}}}
\newcommand{\Ffield}{\ensuremath{\mathbb{F}_{\p}}}
\newcommand{\xcoord}{$x$-coordinate\xspace}
\newcommand{\xcoords}{$x$-coordinates\xspace}


\begin{document}

%don't want date printed
\date{}

%make title bold and 14 pt font (Latex default is non-bold, 16 pt)
\title{\Large \bf Proving the complete correctness of TweetNaCl's Curve25519 implementation.%
\thanks{
Date: somewhen in 2018.}
}

%for single author (just remove % characters)
\ifpublic
\author{
{\rm Peter Schwabe}\\
  Digital Security Group, Radboud University, The Netherlands\\
%\~\\
\and
{\rm Beno\^it Viguier}\\
  Digital Security Group, Radboud University, The Netherlands\\
}
\fi


\maketitle

%\thispagestyle{empty}

\subsection*{Abstract}
By using the Coq formal proof assistant with the VST library, we prove the
soundness and correctness of TweetNaCl's Curve25519 implementation.

%NaCl~\cite{BLS12} is an easy-to-use high-security high-speed software library
for network communication, encryption, decryption, signatures, etc.
TweetNaCl~\cite{BGJ+15} is its compact reimplementation.
It does not aim for high speed application and has been optimized for source
code compactness (100 tweets). It maintains some degree of readability in order
to be easily auditable.
TweetNaCl is being used by ZeroMQ~\cite{zmq} messaging queue system to provide
portability to its users.
``TweetNaCl is the
first cryptographic library that allows correct functionality to be verified
by auditors with reasonable effort''~\cite{BGJ+15}

% XXX: TweetNaCl (find real-world use of TweetNaCl?)
% Mega.nz uses tweetnacl-js (as JS port of tweetnacl) for their webclient https://mega.nz/
% Keybase client: https://github.com/keybase/node-nacl/blob/master/lib/tweetnacl.js

One core component of TweetNaCl (and NaCl) is the key exchange protocol X25519~\cite{rfc7748}.
This protocol is being used by a wide variety of applications~\cite{this-that-use-curve25519}
such as SSH~\cite{rfc4253}, Signal Protocol, Tor, Zcash, TLS to establish a shared secret over
an insecure channel.

This library makes use of Curve25519~\cite{Ber06}, a function over a \F{p}-restricted
$x$-coordinate computing a scalar multiplication on $E(\F{p^2})$, where $p$ is
the prime number $\p$ and $E$ is the elliptic curve $y^2 = x^3 + 486662 x^2 + x$.

Originally, the name ``Curve25519'' referred to this keyexchange protocol,
but Bernstein suggested to rename the scheme to X25519 and to use the name
Curve25519 for the underlying elliptic curve~\cite{Ber14}.
We make use of this notation in this paper.

\subheading{Contribution of this paper}


\todo{Proof that TweetNaCl's X25519 code correctly implements math definition from 25519 paper}

\todo{State additional contributions, e.g., extension of EC framework by Bartiza and Strub.}

Implementing cryptographic primitives without any bugs is difficult.
While tests provides with code coverage, they still don't cover 100\% of the
possible input values. In order to get formal guaranties a software meets its
specifications, two methodologies exist.

The first one is by synthesizing a formal specification and generating machine
code by refinment in order to get a software correct by construction.
This approach is being used in e.g. the B-method~\cite{Abrial:1996:BAP:236705},
F*~\cite{DBLP:journals/corr/BhargavanDFHPRR17}, or with Coq~\cite{CpdtJFR}.

However this first method cannot be applied to an existing piece of software.
In such case we need to write the specifications and then verify the correctness
of the implementation.

\subheading{Our Formal Approach.}
Verifying an existing cryptographic library, we use the second approach.
Our method can be seen as a static analysis over the input values coupled
with the formal proof that the code of the algorithm matches its specification.

We use Coq~\cite{coq-faq}, a formal system that allows us to machine-check our proofs.
A famous example of its use is the proof of the Four Color Theorem~\cite{gonthier2008formal}.
The CompCert, a C~compiler~\cite{Leroy-backend} proven correct and sound is being build on top of it.
To prove its correctness, CompCert uses multiple intermediate languages. The first step of CompCert is done by the parser \textit{clightgen}.
It takes as input C code and generates its Clight~\cite{Blazy-Leroy-Clight-09} translation.

Using this intermediate representation Clight, we use the Verifiable Software Toolchain
(VST)~\cite{2012-Appel}, a framework which uses Separation logic~\cite{1969-Hoare,Reynolds02separationlogic}
and shows that the semantics of the program satisfies a functionnal specification in Coq.
VST steps through each line of Clight using a strongest post-condition strategy.
We write a specification of the crypto scalar multiplication of TweetNaCl and using
VST we prove that the code matches our definitions.

Bartzia and Strub wrote a formal library for elliptic curves~\cite{DBLP:conf/itp/BartziaS14}.
We extend it to support Montgomery curves. With this formalization, we prove the
correctness of a generic Montgomery ladder and show that our specification is an instance of it.

\subheading{Related work.}

\todo{Separate verification of existing code from generating proof-carrying code.}

Similar approaches have been used to prove the correctness of OpenSSL HMAC~\cite{Beringer2015VerifiedCA}
and SHA-256~\cite{2015-Appel}. Compared to their work
our approaches makes a complete link between the C implementation and the formal
mathematical definition of the group theory behind elliptic curves.

Using the synthesis approach, Zinzindohou{\'{e}} et al. wrote an verified extensible
library of elliptic curves~\cite{Zinzindohoue2016AVE}. This served as ground work for the
cryptographic library HACL*~\cite{zinzindohoue2017hacl} used in the NSS suite from Mozilla.

Coq does not only allows verification but also synthesis.
Using correct-by-construction finite field arithmetic~\cite{Philipoom2018CorrectbyconstructionFF}
one can synthesize~\cite{Erbsen2016SystematicSO} certified elliptic curve
implementations~\cite{Erbsen2017CraftingCE}. These implementation are now being used in BoringSSL~\cite{fiat-crypto}.

Curve25519 implementations has already been under the scrutiny~\cite{Chen2014VerifyingCS}.
However in this paper we provide a proofs of correctness and soundness of a C program up to
the theory of elliptic curves.

\todo{Add 1-2 sentences about how this compares? Different limitations etc.}

\subheading{Reproducing the proofs.}
To maximize reusability of our results we placed the code of our formal proofs
presented in this paper into the public domain. They are available at XXXXXXX
with instructions of how to compile and verify our proofs.

\subheading{Organization of this paper.}
Section~\ref{sec2-implem} gives the necessary background on Curve25519
implementation in TweetNaCl and provides the specifications we later prove correct.
Section~\ref{sec3-maths} describes our extension of the formal library by Bartzia and Strub.
Section~\ref{sec4-refl} makes the link between the mathematical model and the C implementation.
In this section we also describe some of the techniques we used to speed up some of the proofs.
Section~\ref{sec5-vst} provides with attention points a VST user should be careful
of in order to avoid unnecessary work.


% Five years ago:
% \url{https://www.imperialviolet.org/2014/09/11/moveprovers.html}
% \url{https://www.imperialviolet.org/2014/09/07/provers.html}

% \section{Related Works}
%
% \begin{itemize}
%   \item HACL*
%   \item Proving SHA-256 and HMAC
%   \item \url{http://www.iis.sinica.edu.tw/~bywang/papers/ccs17.pdf}
%   \item \url{http://www.iis.sinica.edu.tw/~bywang/papers/ccs14.pdf}
%   \item \url{https://cryptojedi.org/crypto/#gfverif}
%   \item \url{https://cryptojedi.org/crypto/#verify25519}
%   \item Fiat Crypto : synthesis
% \end{itemize}
%
% Add comparison with Fiat-crypto


\vspace*{1cm}
{\footnotesize \bibliographystyle{acm}
\bibliography{collection}}

\begin{appendix}
\end{appendix}


\end{document}







