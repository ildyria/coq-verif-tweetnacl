\section{Curve25519 implementation}

\subsection{Implementation}

256 bits words are cut into limbs of 16 bits placed into 64 bits signed
containers.
\begin{lstlisting}[language=C]
typedef long long i64;
typedef i64 gf[16];
\end{lstlisting}

\begin{lstlisting}[language=C]
sv A(gf o,const gf a,const gf b) {
  int i;
  FOR(i,16) o[i]=a[i]+b[i];
}

sv M(gf o,const gf a,const gf b) {
  i64 i,j,t[31];
  FOR(i,31) t[i]=0;
  FOR(i,16) FOR(j,16) t[i+j]+=a[i]*b[j];
  FOR(i,15) t[i]+=38*t[i+16];
  FOR(i,16) o[i]=t[i];
  car25519(o);
  car25519(o);
}

sv car25519(gf o) {
  int i;
  i64 c;
  FOR(i,16) {
    o[i]+=(1LL<<16);
    c=o[i]>>16;
    o[(i+1)*(i<15)]+=c-1+37*(c-1)*(i==15);
    o[i]-=c<<16;
  }
}
\end{lstlisting}

\subsection{What need to be proven?}

\textbf{Soundness} and \textbf{Correctness}.

We show that TweetNaCl's code is \textbf{sound} also know as \textit{shape analysis} \ie

\begin{itemize}
\item absence of array out-of-bounds,
\textit{For each array access, VST requires to prove the range.}
\item absence of overflows/underflow.
\textit{for each operation, VST requires to prove that the resulting value are in ranges.}
\end{itemize}

We also show that TweetNaCl's code is \textbf{correct}:

\begin{itemize}
\item Curve25519 is correctly implemented
\item The number representation
\end{itemize}

\subsection{Correctness Theorem}

The soundness is implied by the functionnal definition of the theorem.
